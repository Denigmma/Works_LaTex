% !TeX spellcheck = ru_RU
% !TEX root = vkr.tex

\section*{Заключение}

В рамках данной работы выполнены следующие задачи:
\begin{itemize}
    \item Разработан и реализован модуль с топологией рекуррентной сети для архитектуры GRU.
    \item Создана модель с архитектурой GRU и интегрированы callback-функции для оптимизации процесса обучения, включая раннее прекращение, сохранение лучшей модели и визуализацию результатов.
    \item Подготовлен специальные датасеты для тестирования, а так же добавлена сложно-аппроксимируемая функция \texttt{hardsin}, реализовано добавление шума в датасет для повышения устойчивости модели.
    \item Проведён подбор метрики (MSE) и настройка гиперпараметров, обеспечивающая хорошие результаты и баланс между качеством аппроксимации и вычислительными затратами.
\end{itemize}

Работа на данном этапе продемонстрировала успешную интеграцию архитектуры GRU в пакет DEGANN.

\textbf{Планы на продолжение работы}


Для дальнейшего расширения функциональности системы планируется:
\begin{itemize}
    \item Разработать собственную (кастомную) реализацию слоёв GRU для расширения функциональности настройки сети.
    \item Завершить тестирование архитектуры на различных наборах данных и провести более глубокий анализ результатов, представив их в итоговом эксперименте.
\end{itemize}

\textbf{Код и результаты}


Код, написанный в ходе работы, свободно доступен здесь: \href{https://github.com/Denigmma/degann/tree/RNN_GRU}{https://github.com/Denigmma/degann/tree/RNN_GRU}

