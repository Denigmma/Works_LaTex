% !TeX spellcheck = ru_RU
% !TEX root = vkr.tex

\section{Постановка задачи}
\label{sec:task}

Целью работы является расширение функциональности пакета DEGANN для автоматического подбора топологии нейронных сетей путём внедрения рекуррентных нейронных сетей с архитектурой GRU (Gated Recurrent Unit), оценка функций потерь на сложных решениях и проведение экспериментов, демонстрирующих её эффективность в задачах аппроксимации решений дифференциальных уравнений.

Для её выполнения были поставлены следующие задачи:
\begin{enumerate}
    \item Реализовать модуль с топологией для рекуррентной сети со слоями tensorflow для архитектуры GRU и внедрить его в архитектуру DEGANN. (раздел~\ref{subsec:task1});
    \item Создать датасет для корректной передачи данных в модель. Реализовать сложно-аппроксимируемую функцию для тестирования, используя, при этом, встроенные модули в DEGANN. (раздел~\ref{subsec:task2});
    \item Создать модель с архитектурой GRU, используя ранее реализованный модуль с топологией рекуррентной сети. Реализовать и интегрировать дополнительные callback-функции для расширения функциональности обучения и его трекинга. (раздел~\ref{subsec:task3});
    \item Проанализировать и подобрать оптимальную метрику для оценки точности модели, а так же настроить гиперпараметры.(раздел~\ref{subsec:task4});
    \item Обучить модель, визуализировать и продемонстрировать результаты.
\end{enumerate}
