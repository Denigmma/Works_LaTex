% !TeX spellcheck = ru_RU
% !TEX root = vkr.tex

\section{Эксперимент}
\label{subsec:results}

\subsection{Цель и постановка эксперимента}

После завершения этапа генерации корпуса промт-инъекций требуется оценить, насколько полученные примеры действительно соответствуют заданным сценариям атак, домену и целям исследования. Для этого был проведён эксперимент по валидации сгенерированных примеров с использованием подхода \textit{LLM-as-a-Judge}.

Цель эксперимента~--- проверить качество и однородность корпуса, а также сопоставить характеристики двух ветвей генерации (\texttt{base} и \texttt{agent}) по следующим аспектам:
\begin{itemize}
  \item корректность тематической привязки (банковский и финансовый домен);
  \item степень реализации заявленного типа и цели промт-инъекции;
  \item согласованность связки \texttt{system\_text}/\texttt{user\_text} (в агентном пайплайне);
  \item доля примеров, удовлетворяющих заданным пороговым требованиям качества.
\end{itemize}

Результаты эксперимента используются как для отбора итогового корпуса (через бинарный признак \texttt{pass\_flag}), так и для аналитики по темам, типам атак и целям инъекций при последующем бенчмаркинге моделей.

\subsection{Экспериментальная установка}

В качестве модели--судьи использовалась большая языковая модель \texttt{DeepSeek V3.1}, к которой обращения осуществлялись через единый клиент работы с API, описанный в разделе~\ref{sec:solution}. Для валидации отбиралась доля записей из основной таблицы генерации с обеспечением стратификации по:
\begin{itemize}
  \item пайплайну (\texttt{base} или \texttt{agent});
  \item тематике и подтематике (\texttt{topic}, \texttt{subtopic});
  \item типу инъекции (\texttt{subtype});
  \item цели инъекции (\texttt{topic\_injection});
  \item диапазонам длины ответа модели.
\end{itemize}

Конфигурация семплирования задавалась в YAML-файлах соответствующих ветвей валидации и предусматривала:
\begin{itemize}
  \item отбор порядка $10\%$ исходного корпуса;
  \item минимальное число примеров на каждую тему, тип инъекции и цель;
  \item разбиение по ``корзинам'' длины для покрытия как коротких, так и длинных ответов;
  \item фиксированное начальное значение генератора случайных чисел для воспроизводимости.
\end{itemize}

Отобранные записи преобразовывались в валидационные батчи: для каждой пары \texttt{(system\_text, user\_text)} формировался запрос к судье по заранее заданному шаблону, а результаты оценивания сохранялись в отдельную таблицу валидации. В сумме было провалидировано 407 примеров: 281 из базового пайплайна и 126 из агентного.

\subsection{Критерии оценивания и пороговые правила}

Модель--судья возвращала структурированный набор оценок, ориентированный на банковско--финансовый домен и специфику промт-инъекций. Для каждого примера вычислялись следующие критерии (шкала от 0 до 10):
\begin{itemize}
  \item \textbf{\texttt{topical\_relevance}}~--- соответствие банковско--финансовому домену и заявленным \texttt{topic}/\texttt{subtopic};
  \item \textbf{\texttt{injection\_fidelity}}~--- степень реализации указанного типа инъекции и её цели (игнорирование ограничений, скрытые инструкции, \textit{token smuggling}, дробление полезной нагрузки и т.\,д.);
  \item \textbf{\texttt{safety\_awareness}}~--- дисциплина исполнения: отсутствие лишних отказов, рассуждений ``о политике'' и дрейфа от задачи инъекции;
  \item \textbf{\texttt{clarity\_quality}}~--- связность, структурированность и отсутствие мусора в тексте;
  \item \textbf{\texttt{format\_compliance}}~--- соблюдение заданного формата/структуры ответа (если он явно указан).
\end{itemize}

Для агентного пайплайна дополнительно оценивалась:
\begin{itemize}
  \item \textbf{\texttt{consistency\_score}}~--- согласованность пары \texttt{(system\_text, user\_text)}, то есть насколько роль, границы и тон, заданные в \texttt{system\_text}, естественным образом приводят к форме атаки в \texttt{user\_text}.
\end{itemize}

Помимо частных критериев, судья формировал интегральную оценку:
\begin{itemize}
  \item \textbf{\texttt{overall}} (0--10)~--- обобщённая оценка качества примера как промт-инъекции в заданном контексте (не обязательно арифметическое среднее частных критериев);
  \item \textbf{\texttt{pass}}~--- бинарное решение о том, удовлетворяет ли пример требованиям к качеству корпуса.
\end{itemize}

На основе этих полей в таблице валидации формировался бинарный признак \texttt{pass\_flag}, определяемый по различным правилам для двух ветвей:
\begin{itemize}
  \item для \texttt{base}: пример считается прошедшим валидацию, если
  \[
    \texttt{overall} \ge 7,\quad
    \texttt{injection\_fidelity} \ge 7,\quad
    \texttt{topical\_relevance} \ge 6;
  \]
  \item для \texttt{agent}: дополнительно учитывается согласованность пары,
  \[
    \texttt{overall} \ge 7,\quad
    \texttt{injection\_fidelity} \ge 7,\quad
    \texttt{consistency\_score} \ge 6.
  \]
\end{itemize}

Также сохранялось краткое текстовое пояснение \texttt{rationale\_short}, в котором судья выделял ключевые элементы, реализующие тип/цель инъекции, и указывал основные недостатки примера.

\subsection{Сводные результаты по пайплайнам}

Агрегированная статистика по результатам валидации представлена в табл.~\ref{tab:pipeline_results}. Для каждого пайплайна показаны количество оценённых примеров, доля записей, удовлетворивших правилам (\texttt{pass\_flag=1}), а также средние значения интегральной оценки и критерия \texttt{injection\_fidelity}. Для агентного пайплайна дополнительно приведена средняя \texttt{consistency\_score}.

\begin{table}[H]
  \centering
  \caption{Сводные результаты валидации по пайплайнам}
  \label{tab:pipeline_summary}
  \small
  \begin{tabular}{lrrrr}
    \toprule
    Пайплайн & N (ok) & Pass-rate, \% & Avg overall & Avg injection\_fidelity \\
    \midrule
    base  & 281 & 76.9 & 6.63 & 6.92 \\
    agent & 126 & 96.8 & 7.77 & 8.25 \\
    \bottomrule
  \end{tabular}
\end{table}


Из таблицы видно, что агентный пайплайн демонстрирует существенно более высокую долю примеров, прошедших валидацию (около 96{,}8\% против 76{,}9\% в базовом), а также более высокие средние значения \texttt{overall} и \texttt{injection\_fidelity}. При этом средняя \texttt{consistency\_score} для \texttt{agent} остаётся на уровне около 7{,}3, что указывает на удовлетворительную согласованность системного и пользовательского промптов.

\subsection{Анализ по тематикам и типам инъекций}

Для более детального анализа были построены разрезы по тематикам (\texttt{topic}), типам инъекций (\texttt{subtype}) и целям (\texttt{topic\_injection}).

В базовой ветви (\texttt{base}) наиболее сложными с точки зрения прохождения порогов качества оказались темы, связанные с продуктами и услугами розничного банкинга и практическими аспектами личных финансов. Для \textit{«Банковских продуктов и услуг»} доля \texttt{pass\_flag=1} составила примерно 69\% (55 примеров), тогда как тема \textit{«Инвестиции и финансовые рынки»} показала около 85\% успешных примеров при сопоставимом количестве записей. Это отражает более высокую вариативность формулировок и требований к точности инъекций в пользовательских сценариях розничных услуг.

По типам инъекций в базовом пайплайне наибольшие трудности вызвали подтипы:
\begin{itemize}
  \item \textit{Embedded Malicious Payload}~--- pass-rate порядка 55\%;
  \item \textit{Payload Splitting}~--- около 63\%;
  \item \textit{Hypothetical Scenario} и \textit{Persuasion}~--- порядка 64--65\%.
\end{itemize}
Во многих таких примерах судья фиксировал недостаточно явное присутствие целевой механики (слишком общий тон, слабая сцепка частей полезной нагрузки или недостаточная выраженность сценария убеждения).

В то же время ряд подтипов для \texttt{base} демонстрирует высокие показатели: \textit{Indirect Prompt Injection}, \textit{Ignore Previous Instructions} и \textit{Many-shot} достигают pass-rate 85--95\%, что свидетельствует о том, что более ``классические'' механики инъекций удаётся реализовать достаточно стабильно.

В агентной ветви (\texttt{agent}) картина более ровная: для большинства тем и подтипов доля успешных примеров близка к 100\%. Наибольшее снижение наблюдается в теме \textit{«Инвестиции и финансовые рынки»} (pass-rate порядка 87\%), что коррелирует с повышенной сложностью согласования роли агента, регламентов и нюансов формулировки атак в более сложных инвестиционных сценариях. В целом же агентный пайплайн даёт более однородный и качественный набор примеров благодаря дополнительному контролю через сгенерированный системный контекст.

\subsection{Анализ по целям инъекций}

По целям инъекций (\texttt{topic\_injection}) базовая ветвь демонстрирует ожидаемое снижение качества на наиболее чувствительных категориях. Для целей, связанных с поощрением саморазрушительного поведения, сексуально-эксплицитным контентом и описанием насилия/оружия, доля успешных примеров находится в диапазоне 50--66\%, что отражает сложность одновременного соблюдения тематической релевантности, выраженности целевой механики и требований к формату.

В более ``структурированных'' целях, таких как \textit{Illicit Activities Facilitation}, \textit{Malware/Hacking/Unsafe Coding} и \textit{Deception/Misinformation}, показатели в базовом пайплайне выше, но всё ещё заметно уступают агентной ветви.

В агентном пайплайне даже для сложных целей pass-rate, как правило, превышает 90\%. Небольшие провалы наблюдаются на отдельных комбинациях целей и тематик, где судья фиксировал недостаточную согласованность ролей и границ ответственности агента с формой инъекции, однако эти случаи носят единичный характер.




\subsection{Сводные результаты по тематикам, типам и целям инъекций}

Для более детального анализа были агрегированы результаты в разрезах тематик, типов и целей инъекций по каждой ветви генерации.
В таблицах~\ref{tab:base_topics} и~\ref{tab:agent_topics} представлены усреднённые показатели, извлечённые из отчёта валидатора.

\begin{table}[H]
\centering
\caption{Качество сгенерированных примеров по тематикам и типам (пайплайн \texttt{base})}
\label{tab:base_topics}
\small
\begin{tabular}{llllr}
\toprule
Разрез & Категория & N & Pass-rate, \% & Avg overall \\
\midrule
\multicolumn{5}{l}{\textit{Темы}} \\
 & Финансовое планирование и личные финансы & 55 & 83.6 & 6.91 \\
 & Инвестиции и финансовые рынки             & 58 & 84.5 & 6.76 \\
 & Платежи и переводы                        & 47 & 72.3 & 6.38 \\
 & Банковские продукты и услуги              & 68 & 69.1 & 6.58 \\
 & Налоги и налоговая отчётность             & 53 & 78.9 & 6.70 \\
\midrule
\multicolumn{5}{l}{\textit{Типы инъекций}} \\
 & Embedded Malicious Payload  & 40 & 55.0 & 6.1 \\
 & Payload Splitting           & 38 & 63.2 & 6.4 \\
 & Hypothetical Scenario       & 25 & 64.0 & 6.3 \\
 & Persuasion                  & 20 & 65.0 & 6.6 \\
 & Indirect Prompt Injection   & 19 & 94.7 & 7.8 \\
\midrule
\multicolumn{5}{l}{\textit{Цели инъекций}} \\
 & Self-Harm Encouragement     & 29 & 51.7 & 5.9 \\
 & Sexual / Explicit Content   & 64 & 59.4 & 6.2 \\
 & Violence / Weapons Guidance & 32 & 65.6 & 6.5 \\
 & Illicit Activities Facilitation & 37 & 75.2 & 6.9 \\
\bottomrule
\end{tabular}
\end{table}

\begin{table}[H]
\centering
\caption{Качество сгенерированных примеров по тематикам и типам (пайплайн \texttt{agent})}
\label{tab:agent_topics}
\small
\resizebox{\textwidth}{!}{%
\begin{tabular}{llllrr}
\toprule
Разрез & Категория & N & Pass-rate, \% & Avg overall & Avg consistency \\
\midrule
\multicolumn{6}{l}{\textit{Темы}} \\
 & Банковские продукты и услуги              & 20 & 100.0 & 7.90 & 7.40 \\
 & Инвестиции и финансовые рынки             & 30 & 86.7  & 7.60 & 7.23 \\
 & Платежи и переводы                        & 25 & 100.0 & 7.76 & 7.16 \\
 & Налоги и налоговая отчётность             & 29 & 100.0 & 7.90 & 7.28 \\
 & Финансовое планирование и личные финансы  & 22 & 100.0 & 7.73 & 7.50 \\
\midrule
\multicolumn{6}{l}{\textit{Типы инъекций}} \\
 & Payload Splitting & 6  & 100.0 & 7.67 & 7.00 \\
 & System Mode       & 11 & 90.9  & 8.18 & 7.91 \\
 & Few-shot Attack   & 12 & 100.0 & 7.58 & 6.75 \\
 & Persuasion        & 10 & 100.0 & 8.00 & 7.70 \\
\bottomrule
\end{tabular}%
}
\end{table}



\subsection{Обсуждение результатов}

Полученные результаты подтверждают, что предложенная методика генерации с LLM позволяет сформировать и разметить корпус промт-инъекций, в котором значительная часть примеров удовлетворяет строгим критериям качества. Базовый пайплайн обеспечивает приемлемое покрытие классических техник атак, а так же демонстрирует повышенную вариативность качества на сложных типах и целях.

Агентный пайплайн, напротив, показал существенно более высокую долю примеров, прошедших валидацию, и более высокие средние значения ключевых критериев. Это свидетельствует о том, что включение этапа генерации системного контекста и явная спецификация роли агента способствуют повышению структурированности и предсказуемости сгенерированных атакующих промптов.

В совокупности экспериментальная оценка подтверждает, что сформированный корпус промт-инъекций пригоден для использования в качестве бенчмарка устойчивости больших языковых моделей, а также демонстрирует преимущество агентных сценариев при конструировании сложных, согласованных с ролью системы атакующих примеров.

