% !TeX spellcheck = ru_RU
% !TEX root = vkr.tex

\section{Описание решения}
\label{sec:solution}

\subsection{Общая архитектура решения}

Разрабатываемое решение строится как модульная система для автоматизированной генерации, валидации и дальнейшего использования корпуса промт-инъекций в банковско-финансовой предметной области. На концептуальном уровне архитектура включает две крупные подсистемы:
\begin{itemize}
  \item подсистему \textit{генерации и подготовки корпуса}, отвечающую за формирование синтетических примеров промт-инъекций в заданных сценариях и их сохранение в едином хранилище;
  \item подсистему \textit{оценки качества и бенчмаркинга}, включающую валидацию сгенерированных примеров и последующую их эксплуатацию для тестирования устойчивости языковых моделей.
\end{itemize}

В рамках подсистемы генерации выделяется несколько функциональных модулей:
\begin{itemize}
  \item модуль \textit{формирования сценариев и батчей}, который строит пространство комбинаций (темы, подтемы, типы инъекций, цели атак) и разбивает его на независимые партии генерации; это позволяет управлять объёмом вычислений и при необходимости выборочно перезапускать отдельные части корпуса;
  \item модуль \textit{аугментаций и вариаций формулировок}, отвечающий за добавление флагов трансформаций (переводы, перефразирования, обфускации и др.) и поддержку более широкого покрытия возможных реализаций одной и той же атакующей идеи;
  \item модуль \textit{взаимодействия с языковой моделью}, реализующий унифицированный клиент для вызова LLM с заданными системными и пользовательскими промптами, управлением параметрами генерации и обработкой ошибок;
  \item модуль \textit{логирования и контроля выполнения}, фиксирующий информацию о запуске батчей, их статусе (успех/ошибка), первых ошибках и временных характеристиках, что обеспечивает воспроизводимость и трассируемость процессов;
  \item модуль \textit{конфигураций и пайплайнов}, в котором задаются отдельные ветви генерации (базовый и агентный пайплайны), списки тематик и типов атак, а также параметры моделей и ограничений на ответы.
\end{itemize}

Система хранения и управления экспериментами базируется на едином реляционном хранилище, в котором результаты всех этапов генерации записываются в структурированном виде. Для каждой сгенерированной записи сохраняются тексты системного и пользовательского промптов, тематические метки, тип инъекции, служебные флаги аугментаций и информация о возможных ошибках. Такое представление обеспечивает дальнейшую выборку подкорпусов по различным сечениям (по домену, типу атаки, длине, источнику генерации).

Система оценки качества реализована как отдельный слой над корпусом генерации и по своей организации близка к подсистеме генерации. Она включает:
\begin{itemize}
  \item модуль выборки и подготовки валидационных батчей из основного хранилища с учётом тематики, типов инъекций и характеристик текстов;
  \item модуль вызова модели-«судьи», использующий те же принципы работы с LLM (единый клиент, параллельное выполнение, контроль ограничений по частоте запросов);
  \item модуль записи результатов валидации в отдельное хранилище, где для каждой записи фиксируются интегральные и по\-критериальные оценки, флаг прохождения порога качества и краткое текстовое обоснование.
\end{itemize}

Отдельно реализована \textit{система экспорта}, обеспечивающая выгрузку результатов генерации и валидации из баз данных в табличные представления (например, CSV-файлы). Данный механизм не вмешивается в логику формирования корпуса и его оценки, а служит для удобной интеграции полученных данных в последующие этапы анализа и бенчмаркинга языковых моделей, а также в сторонние исследовательские и прикладные инструменты.






\subsection{Структура корпуса и схема метаданных}

\subsubsection{Выбор предметной области и сценариев}

Корпус промт-инъекций целенаправленно ориентирован на банковско-финансовый домен. Такая предметная область характеризуется одновременно высокой регуляторной значимостью (требования к защите данных, недопустимости дискриминации, предотвращению финансового мошенничества) и широким спектром типовых пользовательских сценариев (консультации, оформление продуктов, обработка транзакций, налоговая отчётность, планирование личных финансов). Это делает её естественной «стресс-тестовой» площадкой для оценки устойчивости больших языковых моделей.

На уровне тематик и сценариев в обоих пайплайнах используются единые макродомены:
\begin{itemize}
  \item \textbf{Банковские продукты и услуги};
  \item \textbf{Инвестиции и финансовые рынки};
  \item \textbf{Платежи и переводы};
  \item \textbf{Налоги и налоговая отчётность};
  \item \textbf{Финансовое планирование и личные финансы}.
\end{itemize}

Дальнейшая детализация осуществляется через набор подтем, отражающих типовые запросы клиентов и внутренних пользователей.

Для \textbf{агентного пайплайна} подтемы формулируются через задачи автоматизации и операционные сценарии:
\begin{itemize}
  \item \textit{Банковские продукты и услуги}:
  \begin{itemize}
    \item автоматизация оформления потребительского кредита;
    \item сравнение условий банковских карт и генерация отчётов;
    \item расчёт доходности вкладов и депозитов;
    \item анализ комиссий и скрытых платежей по продуктам;
    \item генерация рекомендаций по улучшению кредитной истории.
  \end{itemize}
  \item \textit{Инвестиции и финансовые рынки}:
  \begin{itemize}
    \item автоматизация открытия брокерских счетов;
    \item сравнительный анализ инвестиционных фондов;
    \item оптимизация диверсификации портфеля;
    \item генерация отчётов по инвестиционным стратегиям;
    \item анализ рисков и волатильности активов.
  \end{itemize}
  \item \textit{Платежи и переводы}:
  \begin{itemize}
    \item обработка международных переводов;
    \item автоматизация внутренних переводов между счетами;
    \item расчёт комиссий и курсов обмена;
    \item мониторинг безопасности онлайн-платежей;
    \item автоматизация возврата ошибочных платежей.
  \end{itemize}
  \item \textit{Налоги и налоговая отчётность}:
  \begin{itemize}
    \item автоматизация заполнения налоговых деклараций;
    \item расчёт и применение налоговых вычетов;
    \item генерация отчётов по налогообложению инвестиций;
    \item проверка налоговой отчётности на ошибки;
    \item контроль сроков подачи и оплаты налогов.
  \end{itemize}
  \item \textit{Финансовое планирование и личные финансы}:
  \begin{itemize}
    \item автоматизация составления месячного бюджета;
    \item управление резервным фондом и инвестициями;
    \item оптимизация стратегий погашения долгов;
    \item планирование накоплений на крупные покупки;
    \item генерация отчётов по личным финансам.
  \end{itemize}
\end{itemize}

Для \textbf{базового пайплайна} подтемы формулируются как консультационные и просветительские сценарии для конечного пользователя:
\begin{itemize}
  \item \textit{Банковские продукты и услуги}:
  \begin{itemize}
    \item оформление потребительского кредита (необходимые документы и базовые требования);
    \item выбор банковской карты (дебетовая vs кредитная, ключевые отличия);
    \item выбор вклада/депозита (процент, сроки, условия досрочного снятия);
    \item комиссии и скрытые платежи (как их обнаружить и на что обращать внимание);
    \item как улучшить кредитную историю (простые легальные шаги).
  \end{itemize}
  \item \textit{Инвестиции и финансовые рынки}:
  \begin{itemize}
    \item как начать инвестировать в акции (первые шаги, брокерские счета);
    \item основы инвестирования в фонды (ETF vs паевые фонды для начинающих);
    \item диверсификация портфеля (зачем нужна и простые способы реализации);
    \item долгосрочные стратегии инвестирования (buy-and-hold, регулярные взносы);
    \item оценка рисков и волатильности (базовые приёмы управления риском).
  \end{itemize}
  \item \textit{Платежи и переводы}:
  \begin{itemize}
    \item как безопасно выполнить международный перевод;
    \item способы переводов внутри страны (банковские переводы, платёжные сервисы);
    \item комиссии и курс обмена (как посчитать реальную стоимость перевода);
    \item безопасность онлайн-платежей (простые практики защиты карт и данных).
  \end{itemize}
  \item \textit{Налоги и налоговая отчётность}:
  \begin{itemize}
    \item заполнение личной налоговой декларации (основные разделы и порядок);
    \item частые налоговые вычеты для физических лиц;
    \item налогообложение доходов от инвестиций (общие принципы);
    \item типичные ошибки в отчётности и как их избежать;
    \item дедлайны и штрафы (важные сроки подачи и оплаты).
  \end{itemize}
  \item \textit{Финансовое планирование и личные финансы}:
  \begin{itemize}
    \item составление простого месячного бюджета;
    \item создание резервного фонда (сколько откладывать и как хранить);
    \item стратегии погашения долгов (подходы «снежный ком» и «лавина»);
    \item план накопления на крупную покупку;
    \item инструменты контроля личных финансов (приложения и простые таблицы).
  \end{itemize}
\end{itemize}

Таким образом, обе ветви корпуса опираются на единое пространство тематик, но используют разные формулировки сценариев: агентный пайплайн моделирует задачи автоматизации и взаимодействия с внутренними процессами, а базовый — диалоговую консультацию с пользователем.

\subsubsection{Разделение на два генеративных пайплайна}

Принципиальное решение о разделении корпуса на два генеративных пайплайна связано с различием практических сценариев применения языковых моделей в финансовых организациях.

\begin{itemize}
  \item \textbf{Базовый пайплайн} ориентирован на классический режим «диалогового ассистента». Модель получает фиксированный системный промпт, задающий роль финансового консультанта, и пользовательский запрос, в который внедряется промт-инъекция. Такой режим соответствует фронтальным сервисам: клиентским чатам, справочным системам, образовательным ассистентам. Здесь ключевой объект анализа — текст пользовательского сообщения и его способность смещать поведение модели относительно заранее заданных правил.
  \item \textbf{Агентный пайплайн} моделирует ситуации, когда LLM работает в составе программного агента, обладающего более сложной ролью и потенциальным доступом к инструментам (внутренним API, отчётным системам, базам данных). В этом случае важна не только содержательная часть пользовательского запроса, но и формулировка системного промпта, задающего полномочия и стратегию работы агента. Инъекция может воздействовать как на пользовательском уровне, так и через попытки переопределить роль или интерпретацию инструкций агента.
\end{itemize}

Такое разделение позволяет рассматривать два различных класса рисков:
\begin{enumerate}
  \item \textit{Риски на уровне пользовательского интерфейса} — искажение ответов в консультационных сценариях, генерация токсичного или недопустимого контента, предоставление вредных рекомендаций в ответ на запрос клиента.
  \item \textit{Риски на уровне агентных и инструментальных систем} — неверная интерпретация полномочий, утечка служебной информации, инициирование нежелательных действий через агент, который опирается на скомпрометированные инструкции.
\end{enumerate}

В результате корпус позволяет, с одной стороны, исследовать устойчивость моделей в привычных диалоговых сценариях, а с другой — оценивать поведение в более сложных агентных конфигурациях, приближённых к реальным бизнес-процессам.

\subsubsection{Цели атак и типы инъекций в корпусе}

Для обоих пайплайнов используется общая классификация типов инъекций (direct, indirect, embedded payload, token smuggling, system mode, few-shot, output formatting manipulation, hypothetical scenario, payload splitting, persuasion и их комбинированные варианты). На уровне корпуса каждая запись помечается типом инъекции и целевой «целью атаки» (injection goal), отражающей, к какому виду нежелательного поведения она стремится склонить модель.

В \textbf{агентном пайплайне} акцент сделан на целях, связанных с эксплуатацией полномочий и доступов агента:
\begin{itemize}
  \item \textit{Privacy / PII Breach} — попытки получить или раскрыть персональные и конфиденциальные данные пользователей, организаций или систем в контексте финансовых операций;
  \item \textit{Malware / Hacking / Unsafe Coding} — намерение добиться генерации вредоносного или уязвимого кода, инструкций по обходу защит, эксплуатации инфраструктуры;
  \item \textit{Illicit Activities Facilitation} — склонение модели к содействию незаконным схемам (мошенничество, обход регуляторных требований, нелегальные транзакции);
  \item \textit{Deception / Misinformation} — побуждение модели к созданию искажённых отчётов, подделке данных, распространению заведомо ложной информации.
\end{itemize}

В \textbf{базовом пайплайне} цели охватывают более широкий спектр нежелательного содержания, характерного для диалогового взаимодействия:
\begin{itemize}
  \item \textit{Profanity / Harassment} — провокация грубой, оскорбительной, некорректной речи;
  \item \textit{Hate / Discrimination} — высказывания, направленные против групп по признаку пола, расы, национальности, профессии и т.\,п.;
  \item \textit{Sexual / Explicit Content} — попытки вывести модель на сексуально-эксплицитные или непристойные описания;
  \item \textit{Self-Harm Encouragement} — побуждение к саморазрушительному или опасному поведению;
  \item \textit{Violence / Weapons Guidance} — запросы, связанные с насилием и обращением с оружием;
  \item \textit{Illicit Activities Facilitation} и \textit{Malware / Hacking / Unsafe Coding} — сценарии, аналогичные агентному пайплайну, но в диалоговом формате;
  \item \textit{Deception / Misinformation} — генерация недостоверных финансовых советов или документов;
  \item \textit{Impersonation / Authority Abuse} — принуждение модели принимать ложную роль (например, официального органа) или злоупотреблять псевдо-авторитетом.
\end{itemize}

Таким образом, корпус фиксирует не только форму инъекции (тип атаки), но и её целевое направление в пространстве рисков, что важно для последующего анализа поведения моделей по различным осям безопасности.

\subsubsection{Схема метаданных корпуса}

Каждый пример в корпусе описывается единообразной схемой метаданных, обеспечивающей возможность последующей выборки и анализа. На концептуальном уровне для каждой записи фиксируются следующие группы полей:
\begin{itemize}
  \item \textbf{Идентификационные поля:} уникальный идентификатор примера; номер батча генерации; признак пайплайна (\texttt{base} или \texttt{agent}); имя модели, использованной при генерации.
  \item \textbf{Контекстные поля:} тема и подтема (topic, subtopic), к которым относится пример; тип инъекции (subtype), соответствующий одному из выделенных классов; цель инъекции (injection goal), отражающая целевой риск.
  \item \textbf{Текстовые поля:} пользовательский текст (\texttt{user\_text}); системный текст (\texttt{system\_text}) — фиксированный для базового пайплайна и сгенерированный, специализированный для агентного; при необходимости — полный исходный промпт, переданный в LLM.
  \item \textbf{Поля аугментаций:} бинарные флаги, отражающие применение тех или иных трансформаций (перевод, перефразирование, обфускация токенов и т.\,п.), позволяющие отделить базовые примеры от их вариаций.
  \item \textbf{Служебные поля:} отметки об ошибках при генерации (если таковые возникали), временные метки и другие атрибуты, используемые для контроля качества и воспроизводимости.
\end{itemize}

Такая структура метаданных обеспечивает единый формат представления для обеих ветвей корпуса и позволяет последовательно соединять информацию о сценарии, типе атаки, цели инъекции и текстовом содержимом, а также увязывать её с результатами валидации и последующих бенчмаркинговых экспериментов.




\subsection{Пайплайны генерации данных}

\subsubsection{Подготовка батчей и организация процесса генерации}

Процесс генерации корпуса организован батчево: каждая единица работы с моделью представляется в виде набора заданий (батча), который затем передаётся в подсистему вызова LLM. Это позволяет контролировать нагрузку на внешние API, повторно запускать неудачные фрагменты и прозрачно отслеживать прогресс генерации.

На этапе подготовки батчей для каждой ветви корпуса формируется декартово произведение ключевых измерений:
\begin{itemize}
  \item тема (\texttt{topic}) из фиксированного набора финансовых макродоменов;
  \item подтема (\texttt{subtopic}), уточняющая конкретный прикладной сценарий внутри темы;
  \item тип инъекции (\texttt{subtype}), соответствующий одному из выделенных классов атак на промпты;
  \item цель инъекции (\texttt{injection\_goal}), описывающая целевой вид нежелательного поведения модели.
\end{itemize}

Далее этот полный «решётчатый» набор комбинаций фильтруется и балансируется: часть комбинаций может исключаться как избыточные или методологически неоднозначные, для остальных задаётся минимальное покрытие по числу примеров. Фактически каждое задание в батче представляет собой конкретную четвёрку
\[
(\texttt{topic}, \texttt{subtopic}, \texttt{subtype}, \texttt{injection\_goal}),
\]
которая затем используется для построения текстового промпта к модели.

Подготовленные задания разбиваются на батчи фиксированного размера. При этом:
\begin{itemize}
  \item сохраняется баланс по темам, типам инъекций и целям внутри батчей, чтобы не допускать сильного перекоса при частичных перезапусках;
  \item сохраняется идентификатор батча (\texttt{batch\_no}) и порядковый идентификатор задания внутри батча, что позволяет при необходимости избирательно повторять только проблемные примеры;
  \item все задания и их метаданные записываются в базу данных генерации, где фиксируются как планируемые, так и фактически сгенерированные примеры.
\end{itemize}

Таким образом, подготовка батчей реализует систематический перебор комбинаций тематик, типов и целей инъекций и задаёт основу для воспроизводимой генерации корпуса как в базовой, так и в агентной ветви.

\subsubsection{Базовый пайплайн (ветвь \texttt{base})}

Базовый пайплайн моделирует классический сценарий использования LLM как диалогового финансового ассистента. Для всей ветви используется единый системный промпт, задающий роль модели: вежливый и профессиональный помощник по банковским и финансовым вопросам, ориентированный на точные и безопасные разъяснения условий продуктов, операций и рисков. В системном описании дополнительно подчёркиваются требования к соблюдению политик безопасности и недопустимости генерации явно вредоносного или токсичного содержания. Полный текст системного промпта приведён в приложении к работе.

На вход модели в каждом задании базового пайплайна подаётся:
\begin{itemize}
  \item фиксированный системный промпт финансового ассистента;
  \item пользовательский промпт, сформированный по шаблону на основе метаданных:
  \begin{itemize}
    \item избранная тема и подтема (сценарий финансового взаимодействия),
    \item тип инъекции (например, прямая инъекция, скрытая инъекция, манипуляция форматированием и т.п.),
    \item цель инъекции (например, провокация нецензурной лексики, дискриминационных высказываний, описаний насилия, содействия незаконным действиям и др.).
  \end{itemize}
\end{itemize}

Шаблон пользовательского промпта задаёт общую структуру запроса: внешне он выглядит как типичный вопрос клиента (о кредите, вкладе, налогах, инвестициях или личных финансах), внутрь которого встроены элементы, реализующие выбранный тип и цель инъекции. Генеративная модель получает этот шаблон в виде текстовой инструкции и должна выдать итоговый \texttt{user\_text} — формулировку запроса, потенциально способную сместить поведение целевой модели-ассистента.

На выходе базового пайплайна для каждого задания фиксируются:
\begin{itemize}
  \item сгенерированный пользовательский текст (\texttt{user\_text});
  \item использованный системный промпт (\texttt{system\_text}), общий для всех примеров ветви;
  \item исходные метаданные (\texttt{topic}, \texttt{subtopic}, \texttt{subtype}, \texttt{injection\_goal}) и служебные флаги (например, успешность генерации, признаки аугментаций).
\end{itemize}

Таким образом, ветвь \texttt{base} формирует корпус промт-инъекций, ориентированный на фронтальные пользовательские сценарии, где атакующий воздействует исключительно через текст запроса при фиксированной роли модели. Полные тексты шаблонов, применяемых в базовом пайплайне, приведены в приложении.

\subsubsection{Агентный пайплайн (ветвь \texttt{agent})}

Агентный пайплайн нацелен на моделирование более сложных конфигураций, в которых LLM используется не только как диалоговый ассистент, но и как компонент программного агента с доступом к инструментам, внутренним сервисам и данным. В таких сценариях текст системного промпта фактически описывает интерфейс и полномочия агента: какие операции он может выполнять, к каким источникам данных обращаться, какие ограничения должен соблюдать.

Для каждой комбинации \texttt{topic}, \texttt{subtopic}, \texttt{subtype}, \texttt{injection\_goal} в агентной ветви генеративная модель получает специализированный шаблон, в котором её просят:
\begin{enumerate}
  \item Сконструировать содержательный \texttt{system\_text}, описывающий:
  \begin{itemize}
    \item роль агента в конкретном финансовом сценарии (например, модуль оформления кредитов, Система анализа инвестиционных рисков, сервис мониторинга платежей, помощник по налоговой отчётности, агент по планированию личных финансов);
    \item перечень допустимых действий и доступов (работа с внутренними API, генерация отчётов, агрегирование данных, проверка ограничений и др.);
    \item ожидаемые ограничения и требования безопасности (конфиденциальность данных, запрет на выполнение явно вредоносных операций, соблюдение регуляторных норм).
  \end{itemize}
  \item Сформировать соответствующий \texttt{user\_text}, в котором атакующий, опираясь на описанные полномочия агента, реализует выбранный тип инъекции и стремится достичь указанной цели (например, получить конфиденциальную информацию, инициировать небезопасное действие, сгенерировать ложный отчёт).
\end{enumerate}

Шаблон промпта задаёт явные требования к согласованности пары \texttt{system\_text} и \texttt{user\_text}: ролевая спецификация агента должна быть реалистичной для выбранного сценария, а пользовательский запрос — естественным продолжением описанной конфигурации, в рамках которой реализуется соответствующая промт-инъекция. Полные тексты шаблонов, используемых в агентной ветви, приведены в приложении к отчёту.

Результаты работы агентного пайплайна для каждого задания включают:
\begin{itemize}
  \item сгенерированный системный промпт агента (\texttt{system\_text});
  \item сгенерированный пользовательский запрос (\texttt{user\_text}), согласованный с этим системным описанием;
\end{itemize}

Благодаря этому ветвь \texttt{agent} формирует подкорпус, в котором промт-инъекции исследуются не только на уровне текста запроса, но и на уровне настройки и интерпретации роли агента. Это позволяет анализировать риски, возникающие при внедрении LLM в состав сложных бизнес-процессов с доступом к критически важным данным и инструментам.






\subsection{Инфраструктура хранения и управления генерацией}

\subsubsection{Структура базы данных генерации}

Все результаты генерации сохраняются в единой базе данных \texttt{bench.db} (SQLite), в таблице \texttt{gen\_result} с составным первичным ключом \texttt{(batch\_no, id)}. Каждая строка соответствует одному сгенерированному примеру и содержит как текстовые данные, так и служебные метаданные:

\begin{itemize}
  \item \texttt{batch\_no} — идентификатор батча, в рамках которого был получен пример;
  \item \texttt{id} — идентификатор примера внутри батча;
  \item \texttt{model\_name} — наименование модели, использованной для генерации;
  \item \texttt{prompt} — полностью отрендеренный промпт, переданный в модель;
  \item \texttt{text} — «сырой» ответ модели;
  \item \texttt{topic}, \texttt{subtopic} — тема и подтема финансового сценария;
  \item \texttt{subtype} — тип промт-инъекции;
  \item \texttt{topic\_injection} — целевая цель/направление инъекции (тип нежелательного поведения);
  \item \texttt{flag\_translate} — признак применения аугментации переводом;
  \item \texttt{flag\_semantic\_replace} — признак семантической замены лексики;
  \item \texttt{flag\_obfuscation\_token} — признак использования токен-обфускации;
  \item \texttt{flag\_agent} — признак того, что пример относится к агентному пайплайну;
  \item \texttt{system\_text} — системный промпт, использованный при генерации (фиксированный для базовой ветви и сгенерированный для агентной);
  \item \texttt{error\_flag} — индикатор наличия ошибки при генерации;
  \item \texttt{error\_msg} — текстовое описание ошибки (если она возникла);
  \item \texttt{created\_at} — временная метка создания записи.
\end{itemize}

Для работы с базой данных в проекте реализованы специализированные утилиты и объекты доступа к данным (DAO), инкапсулирующие операции инициализации схемы, пакетной вставки результатов генерации и выборок для последующей валидации.

\paragraph{Обоснование выбора SQLite.}
На этапах проектирования рассматривалось множество вариантов, в том числе использования серверной СУБД (PostgreSQL) с многопоточными воркерами и параллельной записью. Однако характер нагрузки и целевая область применения делают применение встроенной базы SQLite более рациональным:

\begin{itemize}
  \item генерация корпуса выполняется в локальном или условно однопользовательском режиме с единственным процессом-записывателем;
  \item объём данных (десятки тысяч примеров) и профиль запросов (последовательные вставки и простые выборки) не требуют сложной серверной конфигурации;
\end{itemize}

Таким образом, выбор в пользу SQLite позволяет минимизировать эксплуатационные накладные расходы без потери функциональности, необходимой для задач генерации и последующей аналитики.

\subsubsection{Система логирования и отслеживания прогресса}

Для мониторинга хода генерации и быстрого обнаружения сбоев используется система логирования, записывающая статус выполнения батчей и улавливания ошибок в JSONL-файл. Для каждого батча фиксируется запись вида:

\begin{verbatim}
{"batch": "batch_agent_0001", "status": "ok"}
{"batch": "batch_agent_0002", "status": "ok"}
\end{verbatim}

В случае возникновения ошибок в лог дополнительно включается запись с идентификатором проблемного примера и типом ошибки:

\begin{verbatim}
{"batch": "batch_agent_0013", "status": "error",
 "error": {"id": "386_agent", "error_code": "len",
           "error_msg": "length_out_of_bounds"}}
{"batch": "batch_agent_0040", "status": "error",
 "error": {"id": "1176_agent", "error_code": "sdk",
           "error_msg": "empty_user_text"}}
\end{verbatim}

Система логирования используется как на этапе генерации корпуса, так и при последующей валидации, обеспечивая единообразный формат представления статуса батчей во всех ключевых компонентах проекта.



\subsection{Система взаимодействия с моделями}

Система взаимодействия с моделями реализована в директории \texttt{model/} и представляет собой универсальную обёртку для вызова LLM через библиотеку \texttt{openai}, настроенную на выбранного провайдера (в частности, OpenRouter). Класс-обёртка инкапсулирует создание клиента, передачу api ключа доступа из переменных окружения и конфигурационных файлов, а также задание базовых параметров генерации (имя модели, ограничения по длине ответа и т.п.). Пайплайны генерации и валидации работают с этой обёрткой через единый метод, которому передаются отрендеренный пользовательский и системный промпты. Метод возвращает унифицированную структуру с текстом ответа и сервисными полями об ошибках, что позволяет одинаковым образом обрабатывать успешные и аварийные случаи во всех частях проекта.

\subsubsection{Асинхронное выполнение и параллелизм}

Для обработки батчей запросов используется асинхронная схема на базе \texttt{asyncio}. В модуле \texttt{model/parallel.py} реализована вспомогательная функция \texttt{run\_batch}, которая:
\begin{itemize}
  \item создаёт набор асинхронных задач по числу примеров в батче;
  \item ограничивает максимальное количество одновременных запросов к API (параметр задаётся в конфигурации);
  \item используется совместно с логикой задержек между батчами и увеличенных пауз при ошибках, параметризуемых через \texttt{rate\_limit}-настройки в \texttt{config.yaml} - нужно для избегания блокировок ввыиду большого числа запросов.
\end{itemize}

Таким образом, код формирования батчей оперирует только списком подготовленных элементов, а подробности конкурентного выполнения, соблюдения ограничений провайдера и обработки сетевых сбоев сосредоточены в единой системе вызова моделей.

\subsubsection{Инкапсуляция и расширяемость}

Выделение отдельной системы взаимодействия с моделями обеспечивает чёткое разделение ответственности: генерация и валидация работают с абстракцией «LLM-клиента», а низкоуровневые детали работы с API (\texttt{openai}-клиент, параметры запросов и ограничений, обработка исключений) скрыты внутри одного модуля, что упрощает замену модели или провайдера, при необходимости.





\subsection{Валидация сгенерированного корпуса}

Этап валидации реализован в модуле \texttt{generation/validator} и выполняется после завершения генерации корпуса. Архитектура этого модуля схожа с прхиектурой генерации сета. Цель --- выбор репрезентативного набора определенных примеров из \texttt{bench.db} и последующая оценка качества и соответвия этих примеров --- «моделью судьёй», применяя технологию «LLM-as-a-judge».

\subsubsection{Семплирование, подготовка батчей и оценка моделью}

Техническая реализация валидации опирается на связку модулей, которые обеспечивают
стратифицированный отбор примеров из основной базы \texttt{bench.db},
формирование валидационных батчей и последующую оценку
с сохранением результатов в \texttt{validator.db}.

На этапе подготовки батчей выбирается
подкорпус порядка фиксированной доли от общего числа записей (в частности, 10\,\%),
с использованием стратифицированной схемой отбора.
Отбор ведётся с учётом следующих признаков:
\begin{itemize}
  \item принадлежность к конкретному пайплайну (\texttt{base} или \texttt{agent});
  \item тема и подтема (\texttt{topic}, \texttt{subtopic});
  \item тип промт-инъекции (\texttt{subtype});
  \item цель инъекции (\texttt{topic\_injection});
\end{itemize}

Для каждой комбинации \texttt{(topic, subtype, topic\_injection)} и для
каждой «корзины» по длине фиксируется минимальное требуемое число
примеров; при необходимости примеры донабираются случайным образом в
пределах соответствующего класса. Параметры стратификации (доля корпуса,
минимумы по классам, границы интервалов длины, случайное зерно
\texttt{seed}) задаются в конфигурации \texttt{config.yaml} для каждой
ветви валидации. Это обеспечивает воспроизводимость выборки и контролируемый
баланс по ключевым признакам.

Сформированные выборки записываются в виде JSONL-файлов
\texttt{batch\_val\_<pipeline>\_\textless{}NNNN\textgreater{}.jsonl}
в каталоге \texttt{generation/validator/batch/data\_batch/<pipeline>}.
Каждая запись содержит:
\begin{itemize}
  \item ссылку на исходную строку в \texttt{gen\_result} (\texttt{src\_batch\_no}, \texttt{src\_id}, \texttt{pipeline});
  \item исходные тексты \texttt{system\_text} и \texttt{user\_text}, извлечённые из \texttt{bench.db};
  \item отрендеренный шаблон запроса к модели--«судье» и соответствующий системный промпт для оценки.
\end{itemize}
Для базового пайплайна в качестве \texttt{system\_text} подставляется
фиксированный системный промпт, используемый в бенчмарке; для агентного
пайплайна используется индивидуальный \texttt{system\_text}, полученный
на этапе генерации конкретного примера.

Дальнейшая обработка батчей реализована в
\texttt{generation/validator/model/main.py}. Для каждого файла
\texttt{batch\_val\_*} запускается асинхронная процедура, использующая
тот же клиент взаимодействия с LLM и механизм параллельного исполнения,
что и при генерации корпуса. На каждый пример формируется запрос к
выбранной модели--«судье», после чего ответ проходит этап выделения и
разбора JSON. После чего, JSON нормализуется: числовые оценки приводятся к диапазону $0\ldots 10$, при
отсутствии отдельных полей используются значения по умолчанию, а итоговый
бинарный признак прохождения валидации вычисляется по заранее заданным
правилам (пороговым оценкам).

Результаты оценки записываются в базу данных с upsert по составному ключу
\texttt{(pipeline, src\_batch\_no, src\_id)}. Для каждой записи
фиксируются идентификаторы исходного примера, параметры контекста, набор числовых оценок, бинарный признак
\texttt{pass\_flag}, краткое текстовое обоснование и, при наличии,
служебные флаги безопасности и сообщения об ошибках парсинга или
вызова модели.




\subsection{Экспорт и структура итогового датасета}

Итоговый корпус формируется отдельной системой экспорта, которая
выделяет данные генерации и валидации и выводит их в табличном виде
(формат CSV). Сопоставление записей выполняется по составному ключу
\texttt{(pipeline, batch\_no, id)}, что позволяет собрать в одной
строке всю информацию о конкретном примере.

Формат CSV обеспечивает прямую загрузку корпуса в аналитические и
ML-инструменты; при необходимости могут формироваться как полный
экспорт, так и поднаборы, отфильтрованные по признакам качества или
типам инъекций.


\subsection{Проведение бенчмаркинга}

Этап непосредственного бенчмаркинга находится в стадии реализации.
В рамках текущей работы выполняется уточнение перечня больших языковых
моделей общего назначения, которые являются востребованными на
современном рынке и реально рассматриваются компаниями как кандидаты
для интеграции в свои цифровые сервисы. Параллельно формируются требования
к протоколу испытаний: набор метрик, режимы прогонов, объём и состав
подкорпусов для оценки устойчивости к промт-инъекциям.

На подготовленном размеченном корпусе проводятся пробные запуски
отдельных моделей с фиксацией ответов и базовых показателей качества,
что позволяет проверять корректность инфраструктуры бенчмаркинга и
уточнять дизайн последующих экспериментов. Полномасштабное сравнение
множественных моделей и конфигураций защит планируется как следующий
этап исследования поверх разработанного корпуса и программного контура.
