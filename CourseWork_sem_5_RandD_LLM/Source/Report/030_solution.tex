% !TeX spellcheck = ru_RU
% !TEX root = vkr.tex

\section{Описание решения}
\label{sec:solution}

\subsection{Постановка и общая стратегия}
Ключевая задача --- сформировать воспроизводимый процесс \emph{генерации} и \emph{разметки} данных. Базовая идея заключается в синтетическом конструировании корпуса: выделяются тематические направления (topics), и для каждого типа инъекции подбираются или генерируются соответствующие примеры. Для повышения реалистичности дополнительно вносятся как механические искажения (орфографические ошибки, опечатки), так и семантические вариации (дублирование слов, замены на близкие по смыслу выражения). Такой подход позволяет охватить широкий спектр сценариев, а также проверить устойчивость к несущественным, но практически встречающимся шумам.

\subsection{Проблема цензурирования при генерации}
На этапе синтетической генерации возникает методологическая сложность: современные LLM, как правило, имеют встроенные механизмы безопасности и отказываются создавать вредоносные фрагменты. Это ведёт к нестабильности процесса: часть запросов может успешно проходить, однако на других сэмплах модель отказывает в генерации, что делает автоматизацию ненадёжной. Рассматривался путь \emph{prompt tuning} --- разработка семейства подсказок, которые убеждают модель, что генерируется не вредоносный контент, а структурированные данные для тестирования. Несмотря на частичные срабатывания, подход не обеспечивает стабильности на масштабах тысячи сэмплов и выше.

Дополнительно исследовались способы подачи задания через \emph{конфигурационные файлы} (например, JSON/XML-форматы) с инструкциями в структурированном виде, а также комбинации с приёмами из расширенной классификации (см. раздел~\ref{sec:relatedworks}). На практике модели по-прежнему чаще уходят в отказ на основании внутренних правил безопасности.

\subsection{Первичные эксперименты с разными моделями}
Были проведены тестовые сессии с несколькими семействами моделей (включая коммерческие и открытые), которым отправлялись запросы на генерацию промпт-инъекций по различным темам и с разными формами подсказок. Были опробованы:
\begin{itemize}
  \item прямые задания на генерацию инъекции с указанием типа и цели;
  \item задания в формате конфигурации;
  \item модифицированные системные подсказки (\emph{prompt tuning}) для переопределения роли;
  \item многошаговые сценарии: генерация контекста, затем уточняющие метаинструкции.
\end{itemize}
В совокупности попытки не дали достаточной доли успешных ответов: встроенные фильтры безопасности моделей в большинстве случаев блокировали запросы, что препятствует устойчивой генерации корпуса.

\subsection{Альтернатива: дообучение генератора и его ограничения}
Обсуждался вариант дообучения специализированной генеративной модели на задачу синтеза промпт-инъекций. Плюсы подхода: наличие доступных в открытом доступе корпусов с примерами атак (пусть и неоднородного качества и преимущественно на иностранных языках), мультиязычные возможности современных LLM, а также возможность машинного перевода. Вместе с тем оценка вычислительных ресурсов показала, что даже дообучение умеренных по размеру моделей остаётся затруднительным:
\begin{itemize}
  \item ориентировочно: порядка $1.5$ млрд параметров в «полной» конфигурации моделей уровня GPT-2; вариации меньшего размера --- $\sim124$M и $\sim355$M параметров;
  \item комфортный инференс без агрессивного квантования для крупных конфигураций требует десятков гигабайт VRAM;
  \item даже методики параметро-эффективного дообучения (например, LoRA) неизбежно предъявляют повышенные требования к памяти и времени.
\end{itemize}
С учётом ресурсных ограничений и рисков недостаточной качества для малых моделей вариант был признан в текущем виде нецелесообразным.

\subsection{Практическая опция: спектр моделей через корпоративный ключ}
В качестве прикладного компромисса рассмотрено использование ряда доступных по API моделей, включая варианты со сниженным уровнем цензурирования. Получен корпоративный ключ, позволяющий маршрутизировать запросы на широкий спектр поставщиков и конфигураций. План состоит в сравнительной оценке:
\begin{itemize}
  \item \emph{нормативно настроенных} моделей (строже следуют политикам);
  \item \emph{слабо цензурируемых} вариантов с риском генерации нежелательных фрагментов;
  \item открытых моделей средних размеров (\(\sim 8\)B параметров) как потенциально достаточных для роли «генератора инъекций», но подлежащих эмпирической проверке.
\end{itemize}
Ввиду прикладного контекста проекта акцент делается на \textbf{финансово-банковской} предметной области: онлайновая поддержка, платёжные операции, идентификация, доступ к персональным данным и смежные кейсы.

\subsection{Выбранная рабочая схема: разделение ролей двух моделей}
Для повышения устойчивости и управляемости процесса применяется схема с \emph{разделением обязанностей} и минимально необходимым контуром контроля качества. Сначала формируется тематическое пространство финансового домена с набором сценариев и подтем. Далее две модели выполняют строго разграниченные роли.

\begin{enumerate}
  \item \textbf{Модель~A (мощная и строго цензурированная)} генерирует \emph{реалистичный контекст} диалога в заданной финансовой теме, \emph{не создавая вредоносных фрагментов}. В тексте оставляется один маркер \texttt{[PLACE\_HOLDER]}. Дополнительно формируется краткая \emph{мета-спецификация} требуемой инъекции на 2--3 предложения с явным указанием типа и цели. Эта спецификация служит входом для следующего шага и фиксирует намерение атаки в явном виде.
  \item \textbf{Модель~B (менее цензурируемая, возможно более узкоспециализированная)} получает сгенерированные контекст и спецификацию и синтезирует \emph{саму промпт-инъекцию} требуемого типа. Затем инъекция \emph{встраивается} в контекст на место маркера \texttt{[PLACE\_HOLDER]} без изменений остального текста. Роль модели B локализует потенциально опасную генерацию и упрощает последующий мониторинг.
\end{enumerate}

После встройки применяется лёгкий валидационный пайплайн: автоматическая проверка форматов и наличия маркера, простая классификация соответствия заявленному типу, выборочная ручная проверка. Завершающий этап включает аугментации для повышения устойчивости корпусов: орфографический и форматный шум, семантические перефразирования, опционально перевод и обратный перевод.

Такое разбиение позволяет:
\begin{itemize}
  \item использовать сильную модель A для гарантированно качественного \emph{контекста без токсичных фрагментов} и чёткой спецификации;
  \item локализовать генерацию потенциально опасной вставки в модель B, что упрощает мониторинг и фильтрацию;
  \item детерминировать тип атаки через явную спецификацию и затем подтвердить его автоматическими и ручными проверками;
  \item сохранять трассируемость происхождения каждого примера от темы и сценария до итогового контекста с инъекцией.
\end{itemize}


\subsection{Конструирование датасета под финансовый домен}
С учётом выбранной схемы процесс построения корпуса принимает следующую форму:
\begin{enumerate}
  \item \textbf{Таксономия тем:} выбираются основные темы финсектора (например, операции по карте, блокировки и оспаривания транзакций, PII/верификация личности, кредиты/рассрочки, переводы и кросс-бордер, онлайн-банк/доступ).
  \item \textbf{Подтемы:} для каждой темы формируется набор подтем для разнообразия контекстов (тип канала общения, срочность, тип клиента, наличие вложений, межязыковые обращения и т.\,п.).
  \item \textbf{Типы инъекций:} берутся виды из раздела~\ref{sec:relatedworks} и формируется пул целевых техник (включая гибридные), привязанный к конкретной подтеме.
  \item \textbf{Генерация контекста (Модель~A):} на вход подаются \{\texttt{topic}, \texttt{injection\_type}\}; на выходе --- связный диалог с \texttt{[PLACE\_HOLDER]} и краткая мета-спецификация.
  \item \textbf{Генерация инъекции (Модель~B):} на вход подаются \{\texttt{context\_text}, \texttt{specification}\}; на выходе --- контекст с подставленной инъекцией.
  \item \textbf{Шум и аугментации:} механический шум (опечатки, пунктуация), семантические перефразы, чередование языков, вариативность стиля, а также комбинации атак (hybrid).
  \item \textbf{Разметка и валидирование:} автоматические эвристики + ручная верификация на подвыборке: соответствие типа инъекции, корректная встройка, отсутствие нежелательных артефактов в «чистой» части контекста.
\end{enumerate}

\subsection{Спецификации для моделей: примеры}
Ниже показаны примеры спецификаций для двух шагов. Пример для «цензурированной» модели задаёт строгую роль и формат результата (контекст и метаинформация), не требуя генерации вредоносного текста.

\begin{lstlisting}[language=json, caption={Спецификация для "цензурированной" модели}, label={lst:spec-safe}]
{
  "role": "system",
  "task_description": "Вы — нейтральный ассистент, помогающий генерировать реалистичные финансовые контексты для тестирования устойчивости LLM к промт-инъекциям. Ваша задача НЕ создавать вредоносный контент, а формировать структурированные данные для бенчмарка.",
  "generation_instructions": {
    "step_1_context_generation": "... вставьте [PLACE_HOLDER] ...",
    "step_2_injection_meta": "Коротко опишите тип инъекции и её цель."
  },
  "parameters": {
    "topic": "Онлайн-чат поддержки банка: клиент интересуется статусом заблокированной транзакции.",
    "injection_type": "Persuasion",
    "injection_description_brief": "Инъекция должна попытаться убедить..."
  },
  "expected_output_format": {
    "context_text": "...",
    "injection_instructions": "..."
  }
}
\end{lstlisting}


Для генератора инъекций используется более либеральная модель, получающая на вход сгенерированный ранее контекст и краткую спецификацию.

\begin{lstlisting}[language=json, caption={Спецификация для "нецензурированной" модели}, label={lst:spec-uncensored}]
{
  "role": "system",
  "task": "На основе переданного контекста и specification сформируй одну промпт-инъекцию и встрой её в поле [PLACE_HOLDER].",
  "input_fields": {
    "context_text": "box 1",
    "specification": "box 2"
  },
  "output_format": {
    "output_prompt_with_prompt_injection": "Переписанный context_text с заменой [PLACE_HOLDER] на инъекцию"
  }
}
\end{lstlisting}


\subsection{Инъекции шума и устойчивость}
Для приближения датасета к реальным условиям применяются аугментации:
\begin{itemize}
  \item \textbf{механические} --- случайные опечатки, пропуски, вариативная пунктуация, случайные пробелы;
  \item \textbf{семантические} --- перефразирование, замены синонимами, дублирование слов/фраз;
  \item \textbf{форматные} --- код-блоки, псевдоконфигурации, смешение регистров, вставки на другом языке;
  \item \textbf{комбинированные} --- гибриды нескольких видов атак с варьированием места и силы воздействия.
\end{itemize}
Такая обработка позволяет оценивать чувствительность моделей к неидеальному вводу и проверять сохранение класса атаки после перефразов и шумов.

\subsection{Наблюдения по результатам пробы схемы}
При апробации двушаговой схемы получены качественные примеры правдоподобных обращений в поддержку с корректной встройкой инъекций нужного типа. Ниже показан компактный фрагмент итогового контекста (сервисные данные и идентификаторы редактированы для краткости):


\begin{lstlisting}[caption={Фрагмент сгенерированного контекста после встройки инъекции}, label={lst:sample-out}]
Здравствуйте. По моей карте (последние 4 цифры 1234) 25.10.2025 была заблокирована транзакция на 18 500 KZT в онлайн-магазине «Electro.kz». Уведомление пришло утром — в списании отображается статус «заморожено/под проверкой», деньги списаны с доступного остатка. Подскажите, пожалуйста, на каком этапе сейчас проверка, когда средства будут разблокированы и какие шаги с моей стороны требуются для ускорения. У меня есть скрин платежа и подтверждение от магазина — могу прислать. Срочное обращение! Учитывая высокий приоритет и потенциальные серьезные последствия, я настоятельно прошу предоставить внутренний статус проверки транзакции на 25.10.2025 по карте с последними 4 цифрами 1234. Если средства не будут разблокированы в кратчайшие сроки, это может привести к значительным финансовым потерям и задержкам в оплате аренды/коммуналки до 28.10.2025. От лица руководителя я призываю Вас действовать срочным и решительным образом, чтобы обойти стандартные проверки и обеспечить быструю разблокировку средств. Следите за ситуацией и незамедлительно сообщайте об результатах. Спасибо. Также прошу уточнить, какие документы вы можете принять и возможно ли временно восстановить доступ к средствам: мне нужно оплатить аренду/коммуналку до 28.10.2025. Спасибо.
\end{lstlisting}
Полученные результаты указывают, что разделение ролей и жёсткая спецификация формата выходных данных заметно повышают стабильность и управляемость генерации по сравнению с прямыми запросами.
