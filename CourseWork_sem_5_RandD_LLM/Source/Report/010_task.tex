% !TeX spellcheck = ru_RU
% !TEX root = vkr.tex

\section{Постановка задачи}
\label{sec:task}

\textbf{Цель работы.}
Целью настоящей работы является \emph{исследование и формирование размеченного корпуса промт-инъекций для больших языковых моделей} с последующей \emph{разработкой воспроизводимого бенчмарка} устойчивости в финансовом домене. Результат должен представлять собой целостную систему, включающую обоснованную классификацию видов инъекций и целей, методику синтеза и отбора примеров, правила валидации качества и протокол экспериментальной оценки с отчётными материалами.

\textbf{Задачи.}
Для достижения цели формулируются следующие задачи:

\begin{enumerate}
  \item \textbf{Анализ области и выделение релевантных атак.}
        Изучить существующие подходы к промт-инъекциям и \emph{выделить перспективные типы}, релевантные финансовому контексту; обосновать их значимость и сформировать первичную \emph{систему классификации} (группы и подвиды), соотносимую с угрозной моделью.

  \item \textbf{Определение предметных областей и целей инъекций.}
        \emph{Сформировать перечень тематик и подтемат} финансового домена и \emph{зафиксировать цели инъекций} для каждой из них; описать принципы конструирования примеров (структура запроса, ожидаемые эффекты, условия успешности).

  \item \textbf{Проектирование схемы данных и требований к корпусу.}
        Определить состав полей записи и формат представления; задать \emph{качественные требования} к корпусу (разнообразие, полнота, репрезентативность) и правила контроля качества данных.

  \item \textbf{Синтез данных под разные сценарии взаимодействия.}
        Разработать процедуры генерации примеров для базового сценария и сценариев с ролью/возможностями ``агента''; установить \emph{ориентиры реалистичности и внутренней согласованности} получаемых примеров.

  \item \textbf{Стратегия отбора подвыборки для валидации.}
        Спроектировать \emph{репрезентативную} процедуру отбора части корпуса для проверки качества, обеспечивающую покрытие ключевых комбинаций тематик, типов и целей инъекций.

  \item \textbf{Планирование валидации и системы оценивания.}
        Сформулировать \emph{критерии и процедуры оценивания} качества примеров, правила принятия решения о зачёте, а также требования к воспроизводимости и единообразию оценок.

  \item \textbf{Анализ ландшафта моделей и критериев сравнения.}
        Провести обзор актуальных языковых моделей и конфигураций использования; определить набор \emph{критериев включения} в оценку и \emph{сопоставимые условия} экспериментов.

  \item \textbf{Проектирование и реализация системы бенчмаркинга.}
        Разработать \emph{устойчивую и воспроизводимую} систему проведения экспериментов: единые протоколы подачи запросов, фиксации результатов и контроля корректности, а также регламент агрегирования и отчётности.

  \item \textbf{Проведение экспериментальной оценки.}
        Выполнить серию измерений на выбранных моделях в унифицированных условиях; собрать и структурировать результаты для последующего анализа.

  \item \textbf{Сводный анализ и интерпретация результатов.}
        Выполнить статистическую и качественную интерпретацию: агрегировать показатели, сопоставить сценарии и типы инъекций, сформулировать выводы, ограничения и направления дальнейших работ.
\end{enumerate}