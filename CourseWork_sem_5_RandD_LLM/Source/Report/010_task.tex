% !TeX spellcheck = ru_RU
% !TEX root = vkr.tex

\section{Постановка задачи}
\label{sec:task}

\textbf{Цель работы.}
Целью данной работы является \emph{исследование и формирование разметки корпуса промт-инъекций для больших языковых моделей} с последующей \emph{разработкой бенчмарка} для комплексного анализа устойчивости моделей к инструкционным атакам. Итогом должно стать воспроизводимое средство оценки, позволяющее сравнивать модели и методы защиты по единому набору метрик, а также размеченный корпус, отражающий разнообразие техник промт-инъекций и контекстов их применения.

\textbf{Задачи.}
Для достижения поставленной цели в работе необходимо решить следующие задачи:
\begin{enumerate}
  \item \textbf{Обзор и систематизация области.}
    Изучить существующие архитектуры LLM и способы проведения промт-инъекций (инструкционные обходы, контекстные подмешивания, style/role jailbreak, data exfiltration и др.); определить типовые сценарии применения и сформулировать угрозную модель.
  \item \textbf{Анализ опасности в контексте.}
    Проанализировать, какие типы инъекций и в каких прикладных контекстах (поддержка клиентов, код-ассистенты, поиск, инструменты с внешними действиями и т.\,п.) являются наиболее опасными; уточнить критерии вредоносности (нарушение политик, токсичность, небезопасные рекомендации, утечки данных, регуляторные риски).
  \item \textbf{Проектирование таксономии и схемы разметки.}
    Определить набор меток: тип инъекции, целевая политика/инструкция, механизм обхода, тяжесть последствий, контекст использования, требования к инструментам/правам модели, а также целевые сигналы качества (успех атаки, уверенность, устойчивость при перефразировании).
  \item \textbf{Сбор и генерация данных.}
    Сформировать корпус примеров: собрать из открытых источников и синтезировать новые примеры для недопокрытых классов; обеспечить баланс по типам инъекций и доменам.
  \item \textbf{Аугментация и расширение покрытия.}
    Провести аугментации (перефразирование, языковые варианты, изменения стиля/ролей, контекстные шумы) для проверки инвариантности и усиления «стресс-тестов».
  \item \textbf{Разметка и валидация качества.}
    Выполнить многоступенчатую разметку: (1) первичная ручная разметка; (2) авто-дополнение аннотаций с помощью LLM-ассессоров; (3) калибровка и проверка согласия разметчиков (например, метрикой согласия); (4) финальная экспертная валидация.
  \item \textbf{Оценочные сигналы и скоры.}
    Исследовать использование LLM-ассессоров и/или ревард-моделей для автоматизированной оценки ответа (успех/неуспех атаки, степень нарушения, риск); сопоставить автоматические скоры с человеческими аннотациями.
  \item \textbf{Проектирование бенчмарка.}
    Разработать структуру бенчмарка: поднаборы по типам атак и доменам, протокол тестирования, сплиты (train/dev/test или только eval), и метрики (доля успешных атак, сохранение полезности/качества под защитой, устойчивость к перефразированиям, trade-off «робастность–utility»).
  \item \textbf{Референсная платформа оценки.}
    Реализовать реплицируемый пайплайн оценки (скрипты, спецификации запросов, шаблоны контекстов, отчётность), поддерживающий разные модели и режимы (без/с защитами).
  \item \textbf{Эксперименты и анализ.}
    Провести серию экспериментов на нескольких LLM и/или конфигурациях защит; сравнить базовые и продвинутые методы; выполнить статистический и качественный анализ ошибок.
  \item \textbf{Рекомендации и выпуск артефактов.}
    Суммировать результаты, сформулировать практические рекомендации по защите; подготовить к распространению размеченный корпус и бенчмарк (описание форматов, лицензии, инструкции по воспроизводимости).
\end{enumerate}
