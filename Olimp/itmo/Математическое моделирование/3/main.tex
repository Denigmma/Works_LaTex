\documentclass[12pt,a4paper]{article}

\usepackage[T2A]{fontenc}
\usepackage[utf8]{inputenc}
\usepackage[russian]{babel}

\usepackage{amsmath,amssymb}
\usepackage{geometry}
\usepackage{graphicx}
\usepackage{float}

\usepackage{listings}
\usepackage{xcolor}

\geometry{margin=2.2cm}

\lstset{
  basicstyle=\ttfamily\small,
  breaklines=true,
  frame=single,
  numbers=left,
  numberstyle=\tiny,
  tabsize=2,
  showstringspaces=false,
  keywordstyle=\color{blue},
  commentstyle=\color{gray},
  stringstyle=\color{teal},
  columns=fullflexible,
  keepspaces=true
}

\title{Модели дискретных сигналов. Задача 3 (отборочный этап)}
\author{}
\date{}

\begin{document}
\maketitle

\section*{Постановка задачи}
На рисунке дан график функции $|f(\omega)|$, где
\[
f(\omega)=\sum_{k=0}^{\infty} a^{k}e^{-i\omega k}.
\]
Известно, что график пересекает ось ординат ($\omega=0$) в точке перегиба.
По графику требуется найти:
\[
|a|,\quad \min |f|,\quad \Re(a),\quad \Im(a)
\]
с точностью до $0.0001$.

\section{Переход к явной формуле}
По определению функция имеет вид
\[
f(\omega)=\sum_{k=0}^{\infty} a_k e^{-i\omega k}.
\]
В данной задаче предполагается, что коэффициенты последовательности имеют вид
\[
a_k = a^k,
\]
где $a\in\mathbb{C}$ — фиксированное комплексное число.
В этом случае функция $f(\omega)$ представляется в виде геометрического ряда:
\[
f(\omega)=\sum_{k=0}^{\infty} (a e^{-i\omega})^k.
\]
При условии $|a|<1$ данный ряд сходится, и его сумма выражается в замкнутом виде:
\[
f(\omega)=\frac{1}{1-ae^{-i\omega}}.
\]

Рассмотрим модуль полученной функции:
\[
|f(\omega)|=\frac{1}{|1-ae^{-i\omega}|}.
\]
Запишем комплексное число $a$ в полярной форме:
\[
a = r e^{i\theta},\qquad r=|a|,\quad \theta=\arg a.
\]
Тогда
\[
|1-ae^{-i\omega}|^2
=
\left|1-re^{i(\theta-\omega)}\right|^2
=
1+r^2-2r\cos(\theta-\omega),
\]
и, следовательно,
\begin{equation}
|f(\omega)|=\frac{1}{\sqrt{1+r^2-2r\cos(\theta-\omega)}}.
\label{eq:mag}
\end{equation}


\section{Использование данных с графика}

\subsection{Максимум $|f|$}
Из \eqref{eq:mag} видно, что максимум $|f|$ достигается при минимуме знаменателя,
то есть при $\cos(\theta-\omega)=1$:
\[
\max |f| = \frac{1}{|1-r|}=\frac{1}{1-r}.
\]
По графику горизонтальная штриховая линия показывает значение максимума:
\[
\max |f| = 2.2.
\]
Отсюда
\[
\frac{1}{1-r}=2.2 \quad\Rightarrow\quad 1-r=\frac{1}{2.2}=\frac{5}{11}
\quad\Rightarrow\quad r=\frac{6}{11}.
\]
Итак,
\[
|a|=r=\frac{6}{11}=0.5454545\ldots
\]

\subsection{Минимум $|f|$}
Минимум $|f|$ соответствует максимуму знаменателя (при $\cos(\theta-\omega)=-1$):
\[
\min |f|=\frac{1}{1+r}.
\]
Подставляя $r=\frac{6}{11}$:
\[
\min |f|=\frac{1}{1+\frac{6}{11}}=\frac{1}{\frac{17}{11}}=\frac{11}{17}
=0.6470588\ldots
\]

\subsection{Условие перегиба в точке $\omega=0$ и определение фазы}
Обозначим
\[
g(\omega)=|f(\omega)|.
\]
По условию задачи график $|f(\omega)|$ пересекает ось ординат в точке перегиба, то есть
\[
g''(0)=0.
\]

Из формулы
\[
|f(\omega)|=\frac{1}{\sqrt{1+r^2-2r\cos(\theta-\omega)}}
\]
получаем
\[
g(\omega)=\left(1+r^2-2r\cos(\theta-\omega)\right)^{-1/2}.
\]
Дифференцируя по $\omega$ и подставляя $\omega=0$, получаем условие перегиба в виде алгебраического уравнения относительно $c=\cos\theta$:
\begin{equation}
r c^2 + (1+r^2)c - 3r = 0.
\label{eq:quad}
\end{equation}

Подставляя найденное ранее значение $r=\frac{6}{11}$ и приводя к целым коэффициентам, получаем:
\[
66c^2 + 157c - 198 = 0.
\]

\section*{Ответ (точность $0.0001$)}
\[
\boxed{|a| = 0.5455},\qquad
\boxed{\min |f| = 0.6471},\qquad
\boxed{\Re(a)=0.4973},\qquad
\boxed{\Im(a)=0.2241}.
\]

\section{График}
\begin{figure}[H]
\centering
\includegraphics[width=0.92\linewidth]{plot.png}
\caption{Данный в условии график зависимости $|f(\omega)|$.}
\end{figure}

\section{Листинг программы (Python)}
\begin{lstlisting}[language=Python]
import math

f_max = 2.2

r = 1.0 - 1.0 / f_max

A = r
B = 1.0 + r*r
C = -3.0 * r

disc = B*B - 4.0*A*C
c1 = (-B + math.sqrt(disc)) / (2.0*A)
c2 = (-B - math.sqrt(disc)) / (2.0*A)

candidates = []
for c in (c1, c2):
    if -1.0 <= c <= 1.0:
        theta = math.acos(c)
        candidates.append((theta, c))

theta, c = min(candidates, key=lambda t: t[0])

f_min = 1.0 / (1.0 + r)

re_a = r * math.cos(theta)
im_a = r * math.sin(theta)

print("abs_a =", "{:.10f}".format(r))
print("min_abs_f =", "{:.10f}".format(f_min))
print("re_a =", "{:.10f}".format(re_a))
print("im_a =", "{:.10f}".format(im_a))

print("abs_a_4dp =", "{:.4f}".format(r))
print("min_abs_f_4dp =", "{:.4f}".format(f_min))
print("re_a_4dp =", "{:.4f}".format(re_a))
print("im_a_4dp =", "{:.4f}".format(im_a))
\end{lstlisting}

\end{document}
