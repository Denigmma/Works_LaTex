\documentclass[12pt,a4paper]{article}

\usepackage[T2A]{fontenc}
\usepackage[utf8]{inputenc}
\usepackage[russian]{babel}

\usepackage{amsmath,amssymb}
\usepackage{geometry}
\usepackage{graphicx}
\usepackage{float}

\usepackage{listings}
\usepackage{xcolor}

\geometry{margin=2.2cm}

\lstset{
  basicstyle=\ttfamily\small,
  breaklines=true,
  frame=single,
  numbers=left,
  numberstyle=\tiny,
  tabsize=2,
  showstringspaces=false,
  keywordstyle=\color{blue},
  commentstyle=\color{gray},
  stringstyle=\color{teal},
  columns=fullflexible,
  keepspaces=true
}

\title{Модели дискретных сигналов. Задача 2 (отборочный этап)}
\author{}
\date{}

\begin{document}
\maketitle

\section*{Постановка задачи}
Известно, что
\[
x_k = A\cos(\omega k),\qquad \omega\in[0,\pi],
\]
и выполнена периодичность
\[
x_{k+N}=x_k \quad \forall k\in\mathbb{Z}.
\]
В файле \texttt{B.txt} дана зашумлённая последовательность $\{y_k\}_{k=0}^{N-1}$:
\[
y_k = x_k + \varepsilon_k,
\]
где $\varepsilon_k$ --- равномерно распределённый шум.

Требуется:
\begin{enumerate}
\item определить $\omega$;
\item найти коэффициент $A$, минимизирующий среднеквадратичную ошибку
\[
E(A)=\frac{1}{N}\sum_{k=0}^{N-1}\left(y_k-x_k\right)^2
=\frac{1}{N}\sum_{k=0}^{N-1}\left(y_k-A\cos(\omega k)\right)^2.
\]
\end{enumerate}
Численные ответы представить с точностью до $0.0001$ и приложить рассуждения, листинг и графики.

\section{Использование условия периодичности}
Так как $x_{k+N}=x_k$ при всех $k$, то
\[
\cos(\omega(k+N))=\cos(\omega k)\quad \forall k.
\]
Достаточно потребовать периодичность аргумента косинуса по $2\pi$:
\[
\omega N = 2\pi m,\quad m\in\mathbb{Z},
\]
откуда
\begin{equation}
\omega=\frac{2\pi m}{N}.
\label{eq:omega}
\end{equation}
С учётом $\omega\in[0,\pi]$ получаем ограничение на $m$:
\[
0\le \frac{2\pi m}{N}\le \pi \quad\Rightarrow\quad 0\le m\le \frac{N}{2}.
\]
Значит, $\omega$ выбирается из конечного набора кандидатов \eqref{eq:omega}.

\section{Оптимальный $A$ при фиксированном $\omega$}
При фиксированном $\omega$ функция ошибки равна
\[
E(A)=\frac{1}{N}\sum_{k=0}^{N-1}\left(y_k-Ac_k\right)^2,\qquad c_k=\cos(\omega k).
\]
Это квадратичная функция по $A$. Найдём минимум:
\[
\frac{dE}{dA}=\frac{2}{N}\sum_{k=0}^{N-1}\left(y_k-Ac_k\right)(-c_k)=0
\]
\[
\Rightarrow\quad \sum_{k=0}^{N-1} y_k c_k = A\sum_{k=0}^{N-1} c_k^2,
\]
откуда
\begin{equation}
A^\ast(\omega)=\frac{\sum_{k=0}^{N-1} y_k \cos(\omega k)}{\sum_{k=0}^{N-1}\cos^2(\omega k)}.
\label{eq:Aopt}
\end{equation}
Далее для каждого допустимого $m$ вычисляются $\omega=\frac{2\pi m}{N}$, затем $A^\ast(\omega)$ по \eqref{eq:Aopt} и значение $E(A^\ast)$.
Выбирается $\omega$, дающее минимальную ошибку.

\section{Численные результаты}
По данным файла получено $N=500$.
Перебор $m=0,1,\dots,250$ дал минимум ошибки при
\[
m=195,\qquad
\omega=\frac{2\pi\cdot 195}{500}=\frac{39\pi}{50}=2.4504422698\ldots
\]
Оптимальный коэффициент
\[
A = 40.7569637721\ldots
\]

\subsection*{Ответ (точность $0.0001$)}
\[
\boxed{\omega = 2.4504},\qquad
\boxed{A = 40.7570}.
\]

\section{Графики}
\begin{figure}[H]
\centering
\includegraphics[width=0.95\linewidth]{task2_fit_plot.png}
\caption{Сравнение зашумлённых данных $y_k$ и аппроксимации $A\cos(\omega k)$ при найденных $\omega$ и $A$.}
\end{figure}

\begin{figure}[H]
\centering
\includegraphics[width=0.95\linewidth]{task2_mse_plot.png}
\caption{Зависимость MSE от $m$ для кандидатов $\omega=\frac{2\pi m}{N}$. Минимум достигается при $m=195$.}
\end{figure}

\section{Листинг программы (Python)}
\begin{lstlisting}[language=Python]
# -*- coding: utf-8 -*-
import numpy as np
import math
import pathlib
import matplotlib.pyplot as plt

path = "B.txt"

y = np.array([float(v) for v in pathlib.Path(path).read_text().split()])
N = len(y)
k = np.arange(N)

best = None

for m in range(0, N // 2 + 1):
    omega = 2.0 * math.pi * m / N
    c = np.cos(omega * k)

    denom = float(np.dot(c, c))
    if denom < 1e-12:
        continue

    A = float(np.dot(y, c) / denom)
    xhat = A * c

    mse = float(np.mean((y - xhat) ** 2))
    cand = (mse, m, omega, A, xhat)
    if best is None or cand[0] < best[0]:
        best = cand

mse, m, omega, A, xhat = best

print("N =", N)
print("m =", m)
print("omega =", omega)
print("A =", A)
print("MSE =", mse)

print("omega_4dp =", "{:.4f}".format(omega))
print("A_4dp =", "{:.4f}".format(A))

plt.figure(figsize=(9, 4))
plt.plot(k, y, label="y_k", linewidth=1.0)
plt.plot(k, xhat, label="A*cos(omega*k)", linewidth=1.0)
plt.xlabel("k")
plt.ylabel("value")
plt.title("Noisy data and fitted cosine")
plt.grid(True, alpha=0.3)
plt.legend()
plt.tight_layout()
plt.savefig("task2_fit_plot.png", dpi=200)
plt.close()

ms = np.arange(0, N // 2 + 1)
mses = np.zeros_like(ms, dtype=float)

for idx, mm in enumerate(ms):
    ww = 2.0 * math.pi * mm / N
    cc = np.cos(ww * k)
    denom2 = float(np.dot(cc, cc))
    if denom2 < 1e-12:
        mses[idx] = np.nan
        continue
    AA = float(np.dot(y, cc) / denom2)
    mses[idx] = float(np.mean((y - AA * cc) ** 2))

plt.figure(figsize=(9, 4))
plt.plot(ms, mses)
plt.xlabel("m")
plt.ylabel("MSE")
plt.title("MSE vs m for omega = 2*pi*m/N")
plt.grid(True, alpha=0.3)
plt.tight_layout()
plt.savefig("task2_mse_plot.png", dpi=200)
plt.close()
\end{lstlisting}

\end{document}
