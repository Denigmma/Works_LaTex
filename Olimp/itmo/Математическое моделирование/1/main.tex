\documentclass[12pt,a4paper]{article}

\usepackage[T2A]{fontenc}
\usepackage[utf8]{inputenc}
\usepackage[russian]{babel}

\usepackage{amsmath,amssymb}
\usepackage{geometry}
\usepackage{graphicx}
\usepackage{float}

\usepackage{listings}
\usepackage{xcolor}

\usepackage{cmap}     % улучшает извлечение текста из PDF
\usepackage{lmodern}  % более стабильные шрифты
\usepackage{upquote}  % правильные ' и ` в листингах

\geometry{margin=2.2cm}

\lstset{
  basicstyle=\ttfamily\small,
  breaklines=true,
  frame=single,
  numbers=left,
  numberstyle=\tiny,
  tabsize=2,
  showstringspaces=false,
  keywordstyle=\color{blue},
  commentstyle=\color{gray},
  stringstyle=\color{teal},
  columns=fullflexible,
  keepspaces=true
}

\title{Модели дискретных сигналов. Задача 1 (отборочный этап)}
\author{}
\date{}

\begin{document}
\maketitle

\section*{Постановка задачи}
Известно, что дискретный сигнал имеет вид
\begin{equation}
x_k = A\sin(\omega k+\varphi),
\label{eq:model}
\end{equation}
где $\omega\in[2\pi,3\pi]$, $\varphi\in\left[0,\frac{\pi}{2}\right]$, а также выполнена периодичность
\begin{equation}
x_{k+N}=x_k \quad \forall k\in\mathbb{Z}.
\label{eq:period}
\end{equation}
В файле (приложение \texttt{A.txt}) дана последовательность чисел $\{x_k\}$; требуется определить параметры $\omega$, $A$, $\varphi$.
Численные ответы представить с точностью до $0.0001$; приложить рассуждения, аналитические выкладки, листинг программы и график.

\section{Исходные данные и выбор периода}
Из файла считывается $L$ чисел (длина файла). Файл содержит несколько повторов одного и того же периода сигнала, то есть значения идут блоками длины $N$ и повторяются.
Фундаментальный период $N$ находится программно как минимальный делитель $L$, при котором блоки длины $N$ совпадают с допуском порядка $10^{-8}$.
Для данного набора данных получено $N=20$.
Дальнейшие вычисления параметров выполняются по одному периоду $\{x_k\}_{k=0}^{N-1}$, то есть используется срез $x[:N]$.

\section{Аналитическая часть}

\subsection{Связь периода и угловой частоты}
Подставим \eqref{eq:model} в условие периодичности \eqref{eq:period}:
\[
A\sin(\omega(k+N)+\varphi)=A\sin(\omega k+\varphi)\quad \forall k.
\]
Так как $\sin(\cdot)$ имеет период $2\pi$, равенство выполняется при
\begin{equation}
\omega N = 2\pi m,\quad m\in\mathbb{Z}.
\label{eq:omega_condition}
\end{equation}
Отсюда:
\begin{equation}
\omega=\frac{2\pi m}{N}.
\label{eq:omega}
\end{equation}
С учётом ограничения $\omega\in[2\pi,3\pi]$ и найденного $N=20$:
\[
2\pi \le \frac{2\pi m}{20} \le 3\pi
\quad\Rightarrow\quad
20 \le m \le 30.
\]
Таким образом, перебираются целые $m=20,21,\dots,30$.

\subsection{Линеаризация и оценивание параметров}
Модель \eqref{eq:model} приводится к линейной форме по параметрам:
\[
A\sin(\omega k+\varphi)
= A\left(\sin(\omega k)\cos\varphi+\cos(\omega k)\sin\varphi\right)
= B\sin(\omega k)+C\cos(\omega k),
\]
где $B=A\cos\varphi$, $C=A\sin\varphi$.
Для фиксированного $\omega$ оцениваются $B$ и $C$ методом наименьших квадратов:
\[
\min_{B,C}\sum_{k=0}^{N-1}\left(x_k-\left(B\sin(\omega k)+C\cos(\omega k)\right)\right)^2.
\]
После нахождения $B$ и $C$ восстанавливаем
\begin{equation}
A=\sqrt{B^2+C^2},\qquad \varphi=\operatorname{atan2}(C,B).
\label{eq:recover}
\end{equation}
Значение $\varphi$ после $\operatorname{atan2}$ приводится к диапазону $[0,2\pi)$ (при необходимости добавлением $2\pi$).
По условию задачи выбирается представитель в $[0,\pi/2]$; в полученном решении $\varphi\approx 0.81$ уже лежит в требуемом интервале.

\subsection{Критерий выбора $m$}
Для каждого $m$ строится приближение $\hat{x}_k$ и вычисляется среднеквадратичная ошибка:
\[
\mathrm{RMSE}=\sqrt{\frac{1}{N}\sum_{k=0}^{N-1}(x_k-\hat{x}_k)^2}.
\]
Выбирается $m$, дающий минимальное значение RMSE.

\section{Численные результаты}
В результате перебора $m\in\{20,\dots,30\}$ минимум ошибки достигается при $m=23$.
Тогда
\[
\omega=\frac{2\pi\cdot 23}{20}=7.225663103256524\ldots
\]
и
\[
A=17.00000000000001\ldots,\qquad \varphi=0.8099999999999954\ldots
\]
Ошибка аппроксимации:
\[
\mathrm{RMSE}\approx 7.845656339423094\cdot 10^{-14}.
\]

\subsection*{Ответ (точность $0.0001$)}
\[
\boxed{\omega = 7.2257},\qquad
\boxed{A = 17.0000},\qquad
\boxed{\varphi = 0.8100}.
\]

\section{График}
\begin{figure}[H]
\centering
\includegraphics[width=0.95\linewidth]{fit_plot.png}
\caption{Сравнение $x_k$ (данные) и $\hat{x}_k=A\sin(\omega k+\varphi)$ на одном периоде ($N=20$).}
\label{fig:fit}
\end{figure}

\section{Листинг программы (Python)}
\begin{lstlisting}[language=Python, upquote=true]
import numpy as np, math, pathlib
import numpy.linalg as la
import matplotlib.pyplot as plt

path = "A.txt"

x = np.array([float(v) for v in pathlib.Path(path).read_text().split()])
L = len(x)

def detect_period(x, tol=1e-8):
    L = len(x)
    for p in range(1, L + 1):
        if L % p != 0:
            continue
        ok = True
        for i in range(0, L - p, p):
            if np.max(np.abs(x[i:i+p] - x[i+p:i+2*p])) > tol:
                ok = False
                break
        if ok:
            return p
    return L

N = detect_period(x)
xN = x[:N]
k = np.arange(N)

m_min = int(math.ceil(1.0 * N))
m_max = int(math.floor(1.5 * N))

best = None
for m in range(m_min, m_max + 1):
    omega = 2 * math.pi * m / N
    M = np.column_stack([np.sin(omega * k), np.cos(omega * k)])

    B, C = la.lstsq(M, xN, rcond=None)[0]
    xhat = M @ np.array([B, C])

    rmse = la.norm(xhat - xN) / math.sqrt(N)
    A = math.hypot(B, C)
    phi = math.atan2(C, B)

    cand = (rmse, m, omega, A, phi, B, C, xhat)
    if best is None or cand[0] < best[0]:
        best = cand

rmse, m, omega, A, phi, B, C, xhat = best

print("N =", N)
print("m =", m)
print("omega =", omega)
print("A =", A)
print("phi =", phi)
print("RMSE =", rmse)

print("omega_4dp =", "{:.4f}".format(omega))
print("A_4dp =", "{:.4f}".format(A))
print("phi_4dp =", "{:.4f}".format(phi))

plt.figure(figsize=(8, 4))
plt.plot(k, xN, marker='o', linestyle='-', label='data x_k')
plt.plot(k, xhat, marker='x', linestyle='--', label='fit A*sin(omega*k+phi)')
plt.xlabel("k")
plt.ylabel("x_k")
plt.title("N={}, m={}, omega={:.6f}, A={:.6f}, phi={:.6f}".format(N, m, omega, A, phi))
plt.grid(True, alpha=0.3)
plt.legend()
plt.tight_layout()
plt.savefig("fit_plot.png", dpi=200)
plt.close()
\end{lstlisting}

\end{document}
