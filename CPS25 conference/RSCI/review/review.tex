% !TEX program = xelatex
\documentclass[a4paper,12pt]{article}

% Кодировка и поддержка кириллицы
\usepackage{fontspec}
\usepackage{polyglossia}
\setdefaultlanguage{russian}
\setotherlanguage{english}

% Основной шрифт
\setmainfont{Times New Roman}
\newfontfamily\russianfont{Times New Roman}

% Моноширинный шрифт с поддержкой кириллицы для \texttt
\setmonofont{Courier New}[Script=Cyrillic]
\newfontfamily\cyrillicfonttt[Script=Cyrillic]{Courier New}

% Поля страницы
\usepackage[a4paper,%
            left=25mm,%
            right=25mm,%
            top=25mm,%
            bottom=25mm]{geometry}

\begin{document}

\begin{center}
    {\LARGE \textbf{Рецензия}}
\end{center}

\vspace{1em}

\noindent
\textbf{Название статьи:} \emph{Реализация и внедрение архитектуры Gated Recurrent Unit в пакет нейросетевой аппроксимации дифференциальных уравнений DEGANN}

\noindent
\textbf{Рецензент:} Гориховский Вячеслав Игоревич, кандидат физико-математических наук, доцент кафедры системного программирования

\noindent
\textbf{РИНЦ ID:} 3556-5885

\vspace{1em}

Рецензируемая статья посвящена расширению пакета нейросетевой аппроксимации дифференциальных уравнений DEGANN за счёт внедрения рекуррентной архитектуры Gated Recurrent Unit. Авторы обосновывают необходимость перехода от классического многослойного перцептрона к GRU-сети для более эффективного захвата последовательных зависимостей при решении задач аппроксимации ДУ и описывают реализацию класса \texttt{TensorflowGRUNet} с детальным изложением его интеграции в общую модульную структуру пакета.

\vspace{0.5em}

В экспериментальной части проведено сравнение скорости сходимости и качества аппроксимации на наборе характерных тестовых функций, демонстрирующее улучшение показателей при сопоставимых вычислительных затратах. Методика экспериментов изложена последовательно, указаны конфигурация оборудования, выбор функций и параметры обучения, а исходный код и данные доступны для проверки результатов.

\vspace{0.5em}

Статья обладает научной новизной, логичной структурой и убедительной экспериментальной частью, достаточной теоретической проработкой и практической значимостью для пользователей DEGANN и исследователей в области нейросетевой аппроксимации дифференциальных уравнений. Представленные результаты обоснованы и воспроизводимы.

\vspace{1em}

\noindent
\textbf{Рекомендация:} принять к публикации.

\vfill

% Дата слева и подпись справа на одном уровне
\noindent
\begin{minipage}[t]{0.45\textwidth}
Дата: \today
\end{minipage}%
\hfill
\begin{minipage}[t]{0.45\textwidth}
\raggedleft
Подпись: \rule{4cm}{0.4pt}
\end{minipage}

\end{document}
