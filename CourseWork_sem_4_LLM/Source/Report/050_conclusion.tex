% !TeX spellcheck = ru_RU
% !TEX root = vkr.tex

\section{Заключение}

В настоящей работе представлена универсальная система автоматизированного парсинга веб-страниц на основе больших языковых моделей. Система включает следующие ключевые компоненты:

\begin{itemize}
    \item \textbf{Определение типа страницы и загрузка контента.} Реализован модуль, определяющий необходимость JS-рендеринга — при выявлении динамического ресурса выполняется рендеринг через Selenium, иначе используется статический HTTP-запрос \cite{RequestsDocumentation, SeleniumDocumentation}.
    \item \textbf{Стратегии очистки HTML-разметки.} Внедрены две стратегии очистки: «полная» очистка, возвращающая только текст, и «лёгкая» очистка, сохраняющая базовую структуру HTML. Они обеспечивают подготовку данных для режимов Structuring и Codegen \cite{Brown2020}.
    \item \textbf{Интеграция с LLM (режимы \emph{Structuring} и \emph{Codegen}).} Организовано формирование JSON непосредственно из очищенного текста (Structuring) и генерация Python-скриптов с функцией \lstinline|parse| для локального выполнения (Codegen) \cite{Dong2022CacheLLM, Kalyan2023}.
    \item \textbf{Механизм кэширования и семантического поиска.} Использованы SQLite для хранения метаданных и ChromaDB с моделью Sentence-BERT для вычисления векторных эмбеддингов запросов, что позволяет при семантическом совпадении повторно использовать ранее сгенерированные скрипты без повторных обращений к LLM \cite{Reimers2019, ChromaDBDocumentation}.
    \item \textbf{Пользовательские интерфейсы.} Созданы три варианта взаимодействия: Gradio-интерфейс для демонстрационного развёртывания, REST-API на FastAPI для программного доступа и веб-frontend с использованием FastAPI, Jinja2 и JavaScript для удобной работы через браузер \cite{FastAPIDocumentation, GradioDocumentation, Jinja2Documentation}.
    \item \textbf{Экспериментальная валидация.} Проведено тестирование на различных сайтах (статических и динамических), измерены временные характеристики всех режимов работы и оценена точность извлечения данных. Эксперименты подтвердили надёжность, высокую точность и приемлемую производительность системы \cite{Dong2022CacheLLM}.
\end{itemize}

Достоинством предложенного подхода является сочетание традиционных методов парсинга и возможностей LLM с кэшированием результатов, что позволяет обеспечить адаптивность к любым структурам веб-страниц и минимизировать затраты на повторные обращения к модели. Система показала устойчивую работу как с простыми HTML-страницами, так и с современными SPA.

Исходный код доступен в публичном репозитории:
\href{https://github.com/Denigmma/Automated_LLM_data_parsing_system}{https://github.com/Denigmma/Automated\_LLM\_data\_parsing\_system}.
