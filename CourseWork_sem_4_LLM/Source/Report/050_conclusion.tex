% !TeX spellcheck = ru_RU
% !TEX root = vkr.tex

\section{Заключение}

В рамках данной работы была разработана и реализована универсальная система автоматизированного парсинга веб-страниц на основе больших языковых моделей. Система включает следующие ключевые компоненты:

\begin{itemize}
    \item \textbf{Определение типа страницы и загрузка контента.} Введён модуль, который проверяет, является ли страница статической или динамической (SPA), и в зависимости от результата выполняет либо статический HTTP-запрос, либо рендеринг через Selenium.
    \item \textbf{Стратегии очистки HTML-разметки.} Реализованы две стратегии очистки: «полная» очистка, возвращающая только текст, и «лёгкая» очистка, сохраняющая базовую структуру HTML. Они обеспечивают подготовку данных как для прямой генерации JSON, так и для генерации кода-парсера.
    \item \textbf{Интеграция с LLM (режимы \emph{Structuring} и \emph{Codegen}).} Созданы механизмы, позволяющие напрямую формировать структурированные данные (JSON) из очищенного текста страницы и генерировать локальные Python-скрипты с функцией \texttt{parse}, которые затем выполняются для извлечения информации.
    \item \textbf{Механизм кэширования и семантического поиска.} Использован SQLite для хранения метаданных о сгенерированных скриптах и ChromaDB со специализированной моделью для вычисления векторных эмбеддингов запросов. Это позволяет при семантическом совпадении повторно использовать ранее сгенерированные скрипты без обращения к LLM.
    \item \textbf{Пользовательские интерфейсы.} Разработаны три варианта взаимодействия с системой: Gradio-интерфейс для быстрого демонстрационного развёртывания, REST-API на FastAPI для программного доступа и веб-frontend с использованием FastAPI, Jinja2 и JavaScript для удобной работы через браузер.
    \item \textbf{Экспериментальная валидация.} Проведены обширные тесты на различных сайтах (включая статические и динамические ресурсы), измерены временные характеристики всех режимов работы и проверена точность извлечения данных. Эксперименты подтвердили надёжность, высокую точность и достаточную производительность разработанной системы.
\end{itemize}

Исходный код доступен в публичном репозитории: \href{https://github.com/Denigmma/Automated_LLM_data_parsing_system}{https://github.com/Denigmma/Automated\_LLM\_data\_parsing\_system}
