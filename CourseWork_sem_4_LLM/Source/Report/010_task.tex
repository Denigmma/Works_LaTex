% !TeX spellcheck = ru_RU
% !TEX root = vkr.tex

\section{Постановка задачи}
\label{sec:task}

Целью настоящей работы является ускорение сбора данных из открытых веб-источников за счёт автоматизации написания парсеров веб-страниц с помощью больших языковых моделей (LLM). Для достижения этой цели решаются следующие задачи:
\begin{enumerate}
    \item Разработка модуля получения и предварительной обработки веб-страницы, способного определять необходимость JS-рендеринга и загружать как статические, так и динамически генерируемые (SPA) ресурсы (см. раздел~\ref{subsec:solution1}).
    \item Реализация стратегий очистки HTML-разметки: «полная» очистка, получающая чистый текст, и «лёгкая» очистка, сохраняющая базовую структуру документа (см. раздел~\ref{subsec:solution2}).
    \item Построение механизма интеграции с большими языковыми моделями, объединяющего генерацию кода парсера (Codegen) и кэширование результатов (SQLite и ChromaDB) для повторного использования без повторных обращений к LLM (см. раздел~\ref{subsec:solution3}).
    \item Проведение экспериментальной валидации разработанного решения: оценка корректности извлечения данных и измерение производительности в режимах «холодного» (cold) и «тёплого» (warm) запуска парсеров, а также сравнение с подходом Structuring (см. раздел~\ref{subsec:results}).
\end{enumerate}
