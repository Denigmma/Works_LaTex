\section{Эксперимент}
\label{subsec:results}

В этом разделе приводятся результаты проверок предложенной системы автогенерации парсеров на основе LLM. Мы измеряли скорость отклика в различных режимах (\emph{Codegen} и \emph{Structuring}), проверяли работу эмбеддингового кэша для «похожих» запросов и оценивали точность извлечения данных на примере пяти реальных сайтов: Gismeteo, СПбГУ, Habr, Яндекс.Финанс и RussianFood.

\subsection{Условия эксперимента}

\paragraph{Оборудование.}
Эксперименты выполнялись на ноутбуке с процессором Intel i5-1240P, встроенным видеочипом Iris Xe Graphics G7 (80 EUs) и 16 ГБ оперативной памяти.

\paragraph{Параметры запуска.}
Для измерения времени отклика был использован FastAPI-сервер, в котором вызов \texttt{run\_agent} обёрнут в замеры времени (\texttt{time.time()}). Время фетчинга страницы (статического) ограничивалось таймаутом 10~с. Динамический рендеринг (Selenium + Headless Chrome) также выполнялся с таймаутом 10~с, разрешение окна — 1920×1080. LLM-модель — \texttt{mistral-large-latest}, температура = 0.0, \texttt{max\_tokens}=4096. Перед каждым запуском Codegen (cold) локальный кэш полностью очищался:
\begin{verbatim}
rm -rf ./.cache/parsers/*
sqlite3 ./.cache/parsers/cache.db "DELETE FROM code_cache;"
chromadb-cli delete-collection code_cache
\end{verbatim}

\paragraph{Тестовые сайты и запросы.}
Использовались пять веб-страниц:
\begin{enumerate}
    \item \url{https://www.gismeteo.ru/weather-sankt-peterburg-4079/} (статический сайт с метеорологическими данными).
    \item \url{https://spbu.ru/} (официальный сайт СПбГУ, частично динамический).
    \item \url{https://habr.com/ru/articles/701798/} (динамический React-SPA).
    \item \url{https://yandex.ru/finance/currencies} (раздел «Валюты» на Яндекс.Финанс).
    \item \url{https://www.russianfood.com/recipes/recipe.php?rid=138699} (интернет-рецепт на сайте RussianFood).
\end{enumerate}
Для каждого сайта сформированы три тестовых запроса, два из которых близки по смыслу (для проверки работы кэша), а третий — «новый», гарантированно не встречавшийся ранее. Конкретные запросы:

\begin{itemize}
    \item \textbf{Gismeteo:}
    \begin{enumerate}
        \item «Текущая температура в Санкт-Петербурге»
        \item «Покажи температуру сейчас в СПб»
        \item «Скорость ветра и влажность воздуха в Петербурге»
    \end{enumerate}
    \item \textbf{СПбГУ:}
    \begin{enumerate}
        \item «Выведи последние новости СПбГУ»
        \item «Какие анонсы опубликованы на главной странице СПбГУ»
        \item «Выпиши адрес СПбГУ»
    \end{enumerate}
    \item \textbf{Habr:}
    \begin{enumerate}
        \item «Дай заголовок статьи и имя автора»
        \item «Как называется статья и кто её написал»
        \item «Выведи дату публикации и количество просмотров»
    \end{enumerate}
    \item \textbf{Яндекс.Финанс:}
    \begin{enumerate}
        \item «Курс доллара США к рублю»
        \item «Какой курс доллара сейчас»
        \item «Курс евро к рублю»
    \end{enumerate}
    \item \textbf{RussianFood:}
    \begin{enumerate}
        \item «Какие продукты мне понадобятся для приготовления?»
        \item «Что нужно для приготовления этого блюда?»
        \item «Напиши пошаговый рецепт как готовить»
    \end{enumerate}
\end{itemize}

\subsection{Исследовательские вопросы}

\begin{description}
    \item[RQ1:] Насколько быстро система возвращает результаты в режиме \emph{Codegen} (cold и warm) и в режиме \emph{Structuring}?
    \item[RQ2:] Насколько эффективно срабатывает эмбеддинговый кэш при «похожих» запросах (CacheHit)?
    \item[RQ3:] Насколько корректно (точно) система извлекает требуемую информацию для разных типов сайтов?
\end{description}

\subsection{Метрики}

\begin{itemize}
    \item \textbf{Временные метрики:}
    \begin{description}
        \item[$T_{\text{codegen\_cold}}$] — время первого запуска \emph{Codegen}-режима (генерация кода + его исполнение) с очищенным кэшом.
        \item[$T_{\text{codegen\_warm}}$] — время повторного запуска \emph{Codegen}-режима (кэш уже содержит сгенерированный парсер).
        \item[$T_{\text{structuring}}$] — время работы \emph{Structuring}-режима (LLM → JSON).
    \end{description}
    \item \textbf{Метрики точности:}
    \begin{description}
        \item[Accuracy] — доля полностью правильных ответов среди всех запросов (1 — результат совпал с эталоном, 0 — иначе). \\
        \emph{Эталоном служили значения, полученные вручную с исходной страницы.}
        \item[CacheHitRate] — доля «вторых» запросов (для пар «похожих»), для которых сработал кэш (CacheHit = true).
    \end{description}
\end{itemize}

\subsection{Протокол измерений}

\begin{enumerate}
    \item Перед серией измерений каждый раз полностью очищался кэш (файлы, SQLite, ChromaDB).
    \item Для каждого сайта и каждого запроса выполнялись:
    \begin{enumerate}
        \item \emph{Codegen} (cold):
        \begin{itemize}
            \item \texttt{parser.parse\_url(..., mode="codegen", regenerate=False)}.
            \item Замеряется $T_{\text{codegen\_cold}}$, фиксируется CacheHit = \texttt{false} и Accuracy.
        \end{itemize}
        \item \emph{Codegen} (warm):
        \begin{itemize}
            \item \texttt{parser.parse\_url(..., mode="codegen", regenerate=False)} ещё раз (кэш не чистится).
            \item Замеряется $T_{\text{codegen\_warm}}$, CacheHit = \texttt{true} и Accuracy.
        \end{itemize}
        \item \emph{Structuring}:
        \begin{itemize}
            \item \texttt{parser.parse\_url(..., mode="structuring", regenerate=False)}.
            \item Замеряется $T_{\text{structuring}}$ и Accuracy.
        \end{itemize}
    \end{enumerate}
\end{enumerate}

\subsection{Результаты}

\begin{sidewaystable}
    \def\arraystretch{1.1}
    \setlength\tabcolsep{0.7em}
    \centering
    \scriptsize
    \caption{Время отклика (в секундах), CacheHit и точность (\emph{Accuracy}) для каждого сайта и запроса}
    \begin{tabular}{p{5cm} c c c c c}
        \toprule
        \multicolumn{1}{l}{Сайт / Запрос} &
        \multicolumn{1}{c}{\textsc{Codegen (cold)}} &
        \multicolumn{1}{c}{\textsc{Codegen (warm)}} &
        \multicolumn{1}{c}{\textsc{Structuring}} &
        \multicolumn{1}{c}{\textsc{CacheHit}} &
        \multicolumn{1}{c}{\textsc{Acc}} \\
        \midrule
        \textbf{Gismетео} & & & & & \\
        «Текущая температура в СПб»                      & 23.28  & 5.61  & —    & $\checkmark$   & 1.00 \\
        «Покажи температуру сейчас в СПб»                & —      & 5.61  & —    & $\checkmark$   & 1.00 \\
        «Скорость ветра и влажность воздуха в Петербурге» & 40.96  & —     & 8.62 & —              & 1.00 \\
        \midrule
        \textbf{СПбГУ} & & & & & \\
        «Выведи последние новости СПбГУ»                  & 40.91  & 6.09  & —    & $\checkmark$   & 1.00 \\
        «Какие анонсы опубликованы на главной странице»    & —      & 6.09  & —    & $\checkmark$   & 1.00 \\
        «Выпиши адрес СПбГУ»                              & 28.67  & —     & 7.34 & —              & 1.00 \\
        \midrule
        \textbf{Habr} & & & & & \\
        «Дай заголовок статьи и имя автора»               & 47.23  & 17.76 & —    & $\checkmark$   & 1.00 \\
        «Как называется статья и кто её написал»           & —      & 17.76 & —    & $\checkmark$   & 1.00 \\
        «Выведи дату публикации и количество просмотров»   & 75.39  & —     & 29.71 & —             & 1.00 \\
        \midrule
        \textbf{Яндекс.Финанс} & & & & & \\
        «Курс доллара США к рублю»                        & 17.59  & 4.61  & —    & $\checkmark$              & 1.00 \\
        «Какой курс доллара сейчас»                       & —      & 4.61  & —    & $\checkmark$   & 1.00 \\
        «Курс евро к рублю»                                & 25.35  & —     & 12.23& —              & 1.00 \\
        \midrule
        \textbf{RussianFood} & & & & & \\
        «Какие продукты мне понадобятся для приготовления?» & 33.04  & 14.38 & —    & $\checkmark$             & 1.00 \\
        «Что нужно для приготовления этого блюда?»         & —      & 14.38 & —    & $\checkmark$   & 1.00 \\
        «Напиши пошаговый рецепт как готовить»             & 48.99  & —     & 32.71& —              & 1.00 \\
        \bottomrule
    \end{tabular}
    \label{tab:times}
\end{sidewaystable}

\subsection{Примеры JSON-ответов}

Ниже приведены примеры выходных данных в формате JSON для каждого тестового запроса. Семантически совпадающие запросы объединены в один листинг.


\paragraph{Gismетео}
\begin{lstlisting}[language=json,breaklines, keepspaces=true]
{
  "query_data": "Текущая температура в Санкт-Петербурге: +11°C\nТемпература по ощущению: +10°C\n"
}
\end{lstlisting}

%\begin{lstlisting}[language=json,breaklines]
%{
%  "query_data": "Скорость ветра:\n- Время: пн, 2 июня 2025\n  Скорость: 2 м/с\n  Направление: ЮЗ\n- Время: 0:00\n  Скорость: 0 м/с\n  Направление: —\n- Время: 3:00\n  Скорость: 0 м/с\n  Направление: —\n- Время: 6:00\n  Скорость: 1 м/с\n  Направление: Ю\n- Время: 9:00\n  Скорость: 3 м/с\n  Направление: Ю\n- Время: 12:00\n  Скорость: 3 м/с\n  Направление: Ю\n- Время: 15:00\n  Скорость: 3 м/с\n  Направление: ЮЗ\n- Время: 18:00\n  Скорость: 0 м/с\n  Направление: —\n"
%}
%\end{lstlisting}

\paragraph{СПбГУ}
\begin{lstlisting}[language=json, breaklines, keepspaces=true]
{
  "query_data": "Новости СПбГУ:\n 1. Ученые СПбГУ стали лауреатами премии правительства Санкт-Петербурга\n  Дата: 30 мая 2025\n  Ссылка: /news-events/novosti/uchenye-spbgu-stali-laureatami-premii-pravitelstva-sankt-peterburga-0\n\n2. К юбилею почетного профессора СПбГУ Игоря Васильевича Мурина\n  Дата: 29 мая 2025\n  Ссылка: /news-events/novosti/k-yubileyu-pochetnogo-professora-spbgu-igorya-vasilevicha-murina\n\n3. Модельный закон, подготовленный при участии юристов СПбГУ, одобрен Межпарламентской ассамблеей СНГ\n  Дата: 27 мая 2025\n  Ссылка: /news-events/novosti/modelnyy-zakon-podgotovlennyy-pri-uchastii-yuristov-spbgu-odobren\n\n4. Материалы ректорского совещания от 5 мая\n  Дата: 5 мая 2025\n  Ссылка: /news-events/novosti/materialy-rektorskogo-soveschaniya-ot-5-maya\n\n"
}
\end{lstlisting}

\begin{lstlisting}[language=json,breaklines, keepspaces=true]
{
  "query_data": "Адрес СПбГУ: © Санкт-Петербургский государственный университет, 199034, Россия, Санкт-Петербург, Университетская набережная, д. 7–9\n"
}
\end{lstlisting}

\paragraph{Habr}
\begin{lstlisting}[language=json,breaklines, keepspaces=true
{
  "query_data": "Название статьи: Как лучше обучать RNN для прогнозирования временных рядов?\nАвтор: Lev_Perla\n"
}
\end{lstlisting}

\begin{lstlisting}[language=json,breaklines, keepspaces=true]
{
  "query_data": "Дата публикации: 26 ноября 2022 в 22:27\nКоличество просмотров: 22842\n"
}
\end{lstlisting}

\paragraph{Яндекс.Финанс}
\begin{lstlisting}[language=json,breaklines, keepspaces=true]
{
  "query_data": "Курс доллара США к рублю: 78,62 руб\nИзменение курса: Цена опустилась на −2,88 руб\nОтносительное изменение: −2,88 руб (3,53%)\nИсточник: ЦБ РФ\n"
}
\end{lstlisting}

\begin{lstlisting}[language=json,breaklines, keepspaces=true]
{
  "query_data": "Курс евро к рублю\nТекущий курс: 89,25 руб\nИзменение за день: −3,59 руб (3,86%)\nОбновлено: 31 мая 2025 г.\n"
}
\end{lstlisting}

\paragraph{RussianFood}
\begin{lstlisting}[language=json,breaklines, keepspaces=true]
{
  "query_data": "Продукты для приготовления:\n- Молоко — 500 мл\n- Яйца — 2–3 шт.\n- Масло растительное — 1 ст. ложка (+для смазывания сковороды)\n- Мука — 200 г\n- Сахар — 1 ст. ложка\n- Соль — 1 щепотка\n- Масло сливочное — 1 ст. ложка (для смазывания)\n"
}
\end{lstlisting}

%\begin{lstlisting}[language=json,breaklines]
%{
%  "query_data": "### Название рецепта\nБлины на молоке: традиционный рецепт\n\n### Ингредиенты\n- Молоко — 500 мл\n- Яйца — 2–3 шт.\n- Масло растительное — 1 ст. ложка\n- Мука — 200 г\n- Сахар — 1 ст. ложка\n- Соль — 1 щепотка\n- Масло сливочное — 1 ст. ложка\n\n### Шаги приготовления\n1. Размешайте яйца с солью и сахаром. Влейте молоко.\n2. Постепенно добавляйте просеянную муку, перемешивая тесто венчиком.\n3. Добавьте растительное масло, перемешайте, оставьте на 15–20 мин.\n4. Разогрейте сковороду, смажьте маслом.\n5. Вылейте тесто, распределите по поверхности, жарьте до румяности.\n6. Подавайте с вареньем или джемом.\n"
%}
%\end{lstlisting}

\subsection{Обсуждение результатов}

\paragraph{RQ1 (скорость).}
Первый запуск в режиме \emph{Codegen} (cold) потребовал от 23.28 с (Gismетео, «Температура») до 75.39 с (Habr, «Дата и просмотры»). Для новых сайтов: на Яндекс.Финанс первый запрос «Курс доллара» занял 17.59 с, «Курс евро» — 25.35 с; на RussianFood первый запрос «Какие продукты…» — 33.04 с, «Напиши пошаговый рецепт» — 48.99 с. Повторный запуск \emph{Codegen} (warm) занял от 4.61 с (Yandex, «Курс доллара») до 17.76 с (Habr). Режим \emph{Structuring} показал времена в диапазоне: 7.34–32.71 с (минимум — СПбГУ, максимум — RussianFood). Таким образом, \emph{Structuring} оказывается быстрее для ряда задач (кроме «Напиши пошаговый рецепт»), а \emph{Codegen (warm)} демонстрирует приемлемую скорость при повторных запросах.

\paragraph{RQ2 (кэширование).}
Для «похожих» запросов вторые запуски \emph{Codegen} дали \emph{CacheHit = да} и были заметно быстрее. Например, на Яндекс.Финанс «Какой курс доллара сейчас» занял 4.61 с вместо 17.59 с. Аналогично на RussianFood «Что нужно для приготовления…» занял 14.38 с вместо 33.04 с. Это подтверждает корректную работу эмбеддингового кэша.

\paragraph{RQ3 (точность).}
Во всех экспериментах система возвращала полностью верные результаты (Accuracy = 1.00). Эталоном служили ручные данные с исходных страниц. Независимо от структуры сайтов (статический Gismетео, полу-динамический СПбГУ, динамический Habr, Yandex с подгрузкой через XHR, RussianFood с последовательностями блоков), модель корректно извлекала требуемую информацию.

\paragraph{Общие замечания.}
\begin{itemize}
    \item Режим \emph{Codegen (cold)} включает накладные расходы на генерацию и компиляцию кода, что отражается в увеличении времени на первом запуске. Однако в \emph{warm}-режиме кэш существенно ускоряет отклик.
    \item Режим \emph{Structuring} чаще оказывается быстрее, когда нужно получить чистый текст или короткую структуру (например, курс валют). При генерации длинных пошаговых инструкций (как на RussianFood) \emph{Structuring} занимает заметное время (32.71 с), но всё равно быстрее, чем повторный \emph{Codegen}.
    \item Добавление новых сайтов подтвердило устойчивость системы: даже при работе с сильно отличающимися HTML-структурами парсеры LLM возвращают корректные JSON-ответы.
\end{itemize}