% !TEX TS-program = xelatex
% !BIB program = bibtex
% !TeX spellcheck = ru_RU
% !TEX root = talk.tex

\documentclass
  [ russian
  , aspectratio=1610 % Для защит онлайн лучше использовать разрешение не 4х3
  ] {beamer}
\usepackage{blkarray}
\usepackage{listings}


\input{preamble.tex}
\makeatletter

\input{pretitle.tex}
% !TeX spellcheck = ru_RU
% !TEX root = vkr.tex

\filltitle{ru}{
    chair              = {Кафедра информатики},
    group              = {23.Б16-мм},
    title              = {Исследование и формирование разметки корпуса промпт-инъекций для больших языковых моделей с последующей разработкой бенчмарка для комплексного анализа устойчивости к инструкционным атакам},
    type               = {practice},
    kind               = {solution},
    author             = {Мурадян Денис Степанович},
    supervisorPosition = {ст. преп. каф. инф. },
    supervisor         = {В.Д. Олисеенко},
    consultantPosition = {Консультант},
    consultant         = {м.н.с. лПИИ ФИЦ РАН А.А. Вяткин},
}


\newcommand{\academicGroup}{\my@title@group@ru}
\newcommand{\advisorChair}{\my@title@chair@ru}
% То, что в квадратных скобках, отображается внизу по центру каждого слайда.
\title[Auto parsing system]{\my@title@title@ru}
% То, что в квадратных скобках, отображается в левом нижнем углу.
\author[\my@title@author@ru]{\my@title@author@ru, группа \academicGroup}
\institute[СПбГУ]{}
\date[8 июня 2025 г.]{}
\newcommand{\supervisor}{\my@title@supervisor@ru}
\newcommand{\supervisorPosition}{\my@title@supervisorPosition@ru}
\newcommand{\consultant}{\my@title@consultant@ru}
\newcommand{\consultantPosition}{\my@title@consultantPosition@ru}
\newcommand{\reviewer}{\my@title@reviewer@ru}
\newcommand{\reviewerPosition}{\my@title@reviewerPosition@ru}
\newcommand{\defenseYear}{\my@title@year@ru}

\makeatother
\begin{document}
{
\setbeamertemplate{footline}{}
% Лого университета или организации, отображается в шапке титульного листа
\begin{frame}
    \includegraphics[width=1.4cm]{figures/SPbGU_Logo.png}
    \vspace{-35pt}
    \hspace{-10pt}
    \begin{center}
        \begin{tabular}{c}
            \scriptsize{Санкт-Петербургский государственный университет} \\
            \scriptsize{\advisorChair}
        \end{tabular}
        \titlepage
    \end{center}

    \btVFill

    {\scriptsize
        % У научного руководителя должна быть указана научная степень
        \textbf{Научный руководитель:}  \supervisorPosition~\supervisor \\
        % Консультанта может и не быть. Должна быть указана должность или ученая степень
        \textbf{Консультант:}  \consultantPosition~\consultant \\
    }
    \makeatother
    \begin{center}
        \vspace{5pt}
        \scriptsize{Санкт-Петербург\\ \defenseYear}
    \end{center}
\end{frame}
}


\begin{frame}{Введение}
    \begin{itemize}
        \item Веб-парсинг: извлечение структурированных данных из веб-страниц
        \item Проблема: сайты бывают статическими и динамическими, для каждого нужен свой парсер и его постоянная поддержка
        \item Идея: автоматизация генерации parser-скриптов с помощью больших языковых моделей (LLM)
    \end{itemize}
\end{frame}

\begin{frame}{Существующие решения и их недостатки}
    \begin{itemize}
        \item \textbf{Статический парсинг (requests + BeautifulSoup)}
            \begin{itemize}
                \item + Лёгкость и скорость запуска
                \item – Не работает с динамическим JS-контентом, требует ручных селекторов
            \end{itemize}
        \item \textbf{Динамический парсинг (Selenium)}
            \begin{itemize}
                \item + Обходит любой JS
                \item – Тоже требует писать и поддерживать скрипты-парсеры под каждую структуру
            \end{itemize}
        \item \textbf{Общий недостаток}: отсутствует единый самонастраивающийся механизм — при изменении верстки ручной код ломается
    \end{itemize}
\end{frame}

\begin{frame}{LLM-подход и семантический кэш}
    \begin{itemize}
        \item \textbf{Structuring}: извлечение ответа на запрос пользователя напрямую из очищенного текста страницы
        \item \textbf{Codegen (двойное применение LLM)}:
            \begin{itemize}
                \item \emph{HintGen}: генерирует подсказки, селекторы и few-shot примеры
                \item \emph{CodeGen}: на основе подсказок формирует Python-скрипт-парсер и выполняет его
            \end{itemize}
        \item \textbf{Проблемы}: дороговизна запросов в LLM и время инференса ответа
        \item \textbf{Решение}: семантический кэш (SQLite + ChromaDB), для одинаковых ссылок и схожих embedding запросов - запросы в LLM не отправляются, а используются ранее кешированные парсеры
    \end{itemize}
\end{frame}

\begin{frame}{Постановка задачи}
    \textbf{Цель проекта:} оптимизировать сбор данных из открытых веб-источников за счёт автоматизации написания парсеров веб-страниц с помощью больших языковых моделей (LLM)

    \vspace{1em}
    \textbf{Задачи:}
    \begin{itemize}
        \item Определять тип страницы (статическая или динамическая)
        \item Реализовать две стратегии очистки HTML (Full / Lite)
        \item Внедрить режимы LLM-парсинга: Structuring и Codegen
        \item Построить семантический кэш (SQLite + ChromaDB)
        \item Разработать интерфейсы: REST API, веб-фронтенд, Gradio
        \item Провести экспериментальное исследование
    \end{itemize}
\end{frame}


\begin{frame}{Архитектура решения}
    \begin{center}
        \resizebox{\textwidth}{!}{%
        \begin{tikzpicture}[
            >=latex,
            node distance=0.8cm and 1.5cm,
            block/.style={
                draw=black,
                fill=blue!15,
                thick,
                rectangle,
                rounded corners,
                minimum height=2cm,
                minimum width=4cm,
                inner sep=5mm,
                align=center,
                font=\bfseries\Large
            },
            arrow/.style={
                ->,
                thick
            }
        ]
            % Основная линия
            \node[block]                       (input)   {Ввод URL\\и запроса};
            \node[block,right=of input]        (detect)  {Детектор\\динамичности};
            \node[block,right=of detect]       (fetch)   {Fetcher\\(requests / Selenium)};
            \node[block,right=of fetch]        (clean)   {Очистка\\(Full / Lite)};
            \node[block,right=of clean]        (dispatch){Диспетчер\\режимов};
            \node[block,right=of dispatch]     (cache)   {Семантический\\кэш};
            \node[block,right=of cache]        (output)  {Вывод JSON};

            % Ветви LLM
            \node[block,above=of dispatch]     (struct)  {Structuring\\(извлечение ответа)};
            \node[block,below=of dispatch]     (codegen) {Codegen\\(HintGen → CodeGen)};

            % Стрелки
            \draw[arrow] (input)    -- (detect)
                          (detect)  -- (fetch)
                          (fetch)   -- (clean)
                          (clean)   -- (dispatch)
                          (dispatch)-- (cache)
                          (cache)   -- (output);

            \draw[arrow] (dispatch) -- (struct) -- (cache);
            \draw[arrow] (dispatch) -- (codegen) -- (cache);
        \end{tikzpicture}
        }
    \end{center}
\end{frame}




\begin{frame}{Результаты экспериментального исследования}
    \begin{table}
        \centering
        \small
        \begin{tabular}{lcccc}
            \toprule
            \textbf{Сайт / Запрос} & \textbf{Codegen (cold)} & \textbf{Codegen (warm)} & \textbf{Structuring} \\
            \midrule
            \textbf{Gismетео} & & & & \\
            «Текущая температура в СПБ»      & 23.28 & 5.61 & 8.62 \\
            \addlinespace
            \textbf{СПбГУ} & & & & \\
            «Выведи последние новости СПбГУ»  & 40.91 & 6.09 & 7.34 \\
            \addlinespace
            \textbf{Яндекс.Финанс} & & & & \\
            «Курс доллара к рублю»           & 17.59 & 4.61 & 12.23 \\
            \bottomrule
        \end{tabular}
    \end{table}
\end{frame}


\begin{frame}[fragile]{Примеры JSON-ответов}
    \begin{block}{Gismетео: «Текущая температура в Санкт-Петербурге»}
\begin{lstlisting}[language=json,breaklines,keepspaces]
{
  "query_data": "Текущая температура в Санкт-Петербурге: +11°C\nТемпература по ощущению: +10°C\n"
}
\end{lstlisting}
    \end{block}

    \begin{block}{Яндекс.Финанс: «Курс доллара США к рублю»}
\begin{lstlisting}[language=json,breaklines,keepspaces]
{
  "query_data": "Курс доллара США к рублю: 78,62 руб\nИзменение курса: −2,88 руб (−3,53%)\nИсточник: ЦБ РФ\n"
}
\end{lstlisting}
    \end{block}
\end{frame}

\begin{frame}{Итоги работы}
    \begin{itemize}
        \item Разработана модульная система для парсинга статических и динамических веб-страниц
        \item Реализованы два режима LLM-генерации парсеров:
            \begin{itemize}
                \item прямое извлечение ответа из текста (Structuring)
                \item генерация Python-скрипта с помощью подсказок (Codegen)
            \end{itemize}
        \item Внедрён семантический кэш для повторного использования сгенерированных скриптов
        \item Предоставлены разные интерфейсы: библиотека, REST API, веб-приложение и Gradio
        \item Экспериментально подтверждена эффективность подхода на реальных сайтах
    \end{itemize}
\end{frame}


\begin{frame}
    \begin{center}
        \includegraphics[width=6cm]{figures/qr.png}
        \vspace{1em}

        \scriptsize{Воспользоваться проектом можно, отсканировав QR-код для перехода на Hugging Face Space}
    \end{center}
\end{frame}

\end{document}

