\documentclass[a4paper,14pt]{article}
\usepackage[utf8]{inputenc}
\usepackage[russian]{babel}
\usepackage{geometry}
%\geometry{left=1in,right=1in,top=1in,bottom=1in}
\geometry{left=1.4cm,right=1.4cm,top=1cm,bottom=1cm}
\usepackage[colorlinks=true, linkcolor=blue, urlcolor=blue, citecolor=blue]{hyperref}
\usepackage{graphicx}
\usepackage{enumitem}
\usepackage{titlesec}
\usepackage{fontawesome5}
\usepackage{setspace}


\titleformat{\section}{\bfseries\large}{\thesection}{1cm}{}
\titleformat{\subsection}{\bfseries\large}{\thesubsection}{1cm}{}

\titlespacing{\section}{0pt}{0.5em}{0.5em}

\setlength{\parindent}{0pt}
\setlength{\parskip}{0.5em}
\setlist{nosep}

\begin{document}

% Заголовок
\begin{center}
    {\Huge \bfseries Денис Mурадян} \\
    \vspace{0.5em}
    {\Large \bfseries Machine Learning Engineer} \\
    \vspace{1em}
    \faPhone~ \href{tel:89052045723}{89052045723} \hspace{1em}
    \faEnvelope~ \href{mailto:muradyan.denis@inbox.ru}{muradyan.denis@inbox.ru} \hspace{1em}
    \faHome~ \href{https://yandex.ru/maps/-/CHRjvNpm}{Санкт-Петербург, Россия} \\
    \vspace{0.5em}
    \faTelegram~ \href{https://t.me/DenisMuradyan}{Telegram (@DenisMuradyan)}
    \faGithub~ \href{https://github.com/Denigmma}{GitHub (Denigmma)} \hspace{1em}
\end{center}

\section*{Навыки}
\hrule
\vspace{0.5em}
\begin{itemize}
    \item \textbf{Языки} Python, Bash, SQL(базовый), C++ (базовый)
    \item \textbf{ML/DL} Classical methods, Neural methods, PyTorch, TensorFlow, HuggingFace Transformers
    \item \textbf{LLM} LangChain, LangGraph, Prompt Engineering, RAG, API - (GigaChat, GPT, Mistral AI, Qwen)
    \item \textbf{NLP} Tokenization, Embeddings, Semantic search (FAISS, BM25)
    \item \textbf{Данные} Pandas, Polars, NumPy, PostgreSQL, SQLite, Vector databases (ChromaDB, FAISS)
    \item \textbf{Визуализация} Matplotlib, Plotly, Seaborn
    \item \textbf{MLOps} FastAPI, Docker, CI/CD, Git, HuggingFace Spaces
\end{itemize}

\section*{Образование}
\hrule
\vspace{0.5em}
\begin{itemize}
%    \item Окончил филиал московской частной школы "Столичный лицей" с углубленным изучением математики, физики и информатики.
    \item \textbf{СПБГУ}, Математико-механический факультет, направление "Искусственный интеллект и наука о данных". 2023–2027 (3 курс)
\end{itemize}

%\section*{Языки}
%\hrule
%\vspace{0.5em}
%    Русский (родной), Английский (Upper-Intermediate)

\section*{Олимпиады}
\hrule
\vspace{0.5em}
\begin{itemize}
    \item \textbf{\href{https://yandex.ru/profi/?etext=}{«Я профессионал»}} (Yandex), трек «Искусственный интеллект» - \textbf{призер} 2024-2025 \href{https://github.com/Denigmma/Practice_LogBook/blob/main/Yandex/Я%20профессионал%20(olimp)/2024-2025/Нормативные%20документы%20и%20диплом/Диплом%20призера%20VIII%20сезона%20ЯП%20ИИ.pdf}{certificate}
    \item \textbf{\href{https://psrs.spbu.ru/}{«PSRS»}} (SPBU), трек «Вычислительные технологии» - \textbf{победитель} 2024-2025 \href{https://github.com/Denigmma/Practice_LogBook/blob/main/SPBU%20olympics%20Petropolitan%20Science%20(Re)Search/2024-2025/Diploma%20winner%2024-25.pdf}{certificate}
    \item \textbf{\href{https://studolymp.gazprom.ru/}{«Газпром»}}, трек «Информационные системы и технологии» - \textbf{призер} 2024-2025 \href{https://github.com/Denigmma/Practice_LogBook/blob/main/Gazprom/Диплом%20Мурадян%20Денис%20Степанович%20по%20профилю%20_Информационные%20системы%20и%20технологии_.pdf}{certificate}
%    \item «Высшая лига» (HSE), трек «ПМИ» - \textbf{участник финала} 2024-2025
%    \item Участник финального этапа олимпиады "Высшая лига" по направлению "Прикладная информатика и математика" 2024-2025
%    \item Международная студенческая олимпиада «Petropolitan Science (Re)Search от СПбГУ», по треку «Вычислительные технологии» 2024-2025 - победитель \href{https://github.com/Denigmma/Practice_LogBook/blob/main/SPBU%20olympics%20Petropolitan%20Science%20(Re)Search/2024-2025/Diploma%20winner%2024-25.pdf}{certificates}
%    \item Всероссийская олимпиада студентов «Я профессионал» по треку «Искусственный интеллект», сезон (2024-2025) - призер \href{https://github.com/Denigmma/Practice_LogBook/blob/main/Yandex/Я%20профессионал%20(olimp)/2024-2025/Нормативные%20документы%20и%20диплом/Диплом%20призера%20VIII%20сезона%20ЯП%20ИИ.pdf}{certificates}
%    \item Всероссийская студенческая олимпиада «Газпром», по треку «Информационные системы и технологии» 2024-2025 - призер \href{https://github.com/Denigmma/Practice_LogBook/blob/main/Gazprom/Диплом%20Мурадян%20Денис%20Степанович%20по%20профилю%20_Информационные%20системы%20и%20технологии_.pdf}{certificates}
\end{itemize}

\section*{Научная деятельность}
\hrule
\vspace{0.5em}
\begin{itemize}
    \item Выступал на \textbf{LVI Международной научной конференции} Control Processes and Stability (CPS25), в Санкт-Петербурге 7 – 11 апреля 2025 года. \\
    \textbf{Тематика:} «Использование рекуррентных нейронных сетей в аппроксимации дифференциальных уравнений»\\
    Статья опубликовалась в сборнике трудов конференции \href{https://www.elibrary.ru/item.asp?id=82486855}{Elibrary}
\end{itemize}

%\section*{Соревнования - Хакатоны}
%\hrule
%\vspace{0.5em}
%\begin{itemize}
%    \item \textbf{Хакатон SPB Tech Run 2024 - 3 место} \\
%    Разработка Интеллектуального чат-бота "Я Здесь Живу" (ЯЗЖ) на основе технологии Large Language Model (LLM) - для генерации ответов, с применением подхода Retrieval-Augmented Generation (RAG) - для эффективного поиска и использования релевантной информации из базы знаний. AI-помощник способен отвечать на разнообразные вопросы жителей о повседневной жизни в Санкт-Петербурге и его районах. Для предоставления ответов на повседневные вопросы, использует открытые данные города.
%
%    Хакатон проводился с 30 ноября по 1 декабря 2024 года, в нём участвовало 80 человек. Наша команда заняла 3 место, продемонстрировав сильный акцент на практическую пользу, техническую проработку и возможность масштабирования проекта.
%    Реализацию бота вы можете посмотреть на \href{https://github.com/Denigmma/Hackathon_SPBTechRun_ILiveHereBot}{GitHub (Hackathon SPBTechRun ILiveHereBot)}
%\end{itemize}


%\section*{Учебно-курсовые работы}
%\hrule
%\vspace{0.5em}
%\begin{itemize}
%\item Внедрение архитектуры рекуррентных нейронных сетей GRU в библиотеку \href{https://github.com/Krekep/degann}{DEGANN}
%
%\item Реализация автоматизированной системы парсинга данных на основе больших языковых моделей \href{https://github.com/Denigmma/Automated_LLM_data_parsing_system}{Automated LLM data parsing system}
%
%\end{itemize}


\section*{Проектная деятельность}
\hrule
\vspace{0.5em}
\begin{itemize}

%\item \textbf{Оптимизированное построение и визуализация графов:} Построение графов с алгоритмом Force-Directed Graph Drawing. Работал над визуализацией графов, минимизируя пересечения рёбер.

\item \textbf{Автоматизированная система парсинга данных на основе LLM}\\ \href{https://github.com/Denigmma/Automated_LLM_data_parsing_system}{GitHub Automated LLM data parsing system}\\
Реализовал LLM-систему парсинга веб-данных по запросу и кэшированием (SQLite, ChromaDB/SBERT). Работал с Mistral AI, занимался prompt-engineering, внедрил auto few-shot систему и семантический поиск для извлечения данных из кэша. Дополнитеольно: проект портировался под агентное решение


\item \textbf{Хакатон SpbTechRun 2024 - 3 место} \\ \href{https://github.com/Denigmma/Hackathon_SPBTechRun_ILiveHereBot}{GitHub (Hackathon SPBTechRun ILiveHereBot)} \\
Разработал LLM-чатбота «Я Здесь Живу» на базе GigaChat для ответов жителям Санкт-Петербурга: RAG-конвейер на LangChain с локальной векторной БД, парсинг открытых городских API, prompt-tuning и логирование; интеграция через Telegram Bot API, развёртывание на FastAPI + Docker.
%    Разработка Интеллектуального чат-бота "Я Здесь Живу" (ЯЗЖ) на основе технологии (LLM) - для генерации ответов, с применением подхода Retrieval-Augmented Generation (RAG) - для эффективного поиска и использования релевантной информации из базы знаний. AI-помощник способен отвечать на разнообразные вопросы жителей о повседневной жизни в Санкт-Петербурге и его районах. Для предоставления ответов на повседневные вопросы, использует открытые данные города.
%    Хакатон проводился с 30 ноября по 1 декабря 2024 года, в нём участвовало 80 человек. Наша команда заняла 3 место, продемонстрировав сильный акцент на практическую пользу, техническую проработку и возможность масштабирования проекта.


\item \textbf{Прогнозирование отказов оборудования в нефтегазовой промышленности} \\
\href{https://github.com/Denigmma/Gazprom_AI_analyzer_equipment_failure}{Gazprom AI analyzer equipment failure}.\\
\textit{Разработано для финального этапа студенческой олимпиады «Газпром»}\\
Создал систему прогнозирования и раннего предупреждения аварийных ситуаций на рекуррентной нейронной сети с GRU по многомерным временным рядам (показаниям с датчиков в реальном времени); AUC 0.9777, Recall 0.9178, Accuracy 0.9125, Precision 0.8490.
Реализовано на FastAPI с real-time dashboard, публикация через LocalTunnel.


%Разработан для заключительного этапа студенческой олимпиады «Газпром» - конкурс проектов.
%Проект представляет собой интеллектуальную систему мониторинга и прогнозирования аварийных ситуаций на
%базе рекуррентной нейронной сети с архитектурой GRU. Модель анализирует временные ряды, поступающие с датчиков
%(давление, температура, вибрация, расход газа и т.д.), и определяет вероятность сбоя или аварии.
%Достигнуты высокие показатели качества: (accuracy = 0.9125, auc = 0.9777, precision = 0.8490, recall = 0.9178). \\[0.3em]
%Система реализована на Python с использованием фреймворка FastAPI (REST API). В процессе работы данные нормализуются,
%подаются на вход модели, а результаты прогнозов отображаются в реальном времени на встроенном dashboard.
%Для удобства развёртывания предусмотрена возможность запуска проекта как локально, так и через LocalTunnel,
%что обеспечивает доступ к сервису из сети Интернет.
%Благодаря учёту временных взаимосвязей, высокой полноте (recall) и возможности оперативной визуализации система эффективно помогает
%предотвращать потенциальные аварии, снижая риски и экономические потери на предприятиях нефтегазовой сферы. \\[0.3em]

\item \textbf{Рекомендательная ситстема новостей} \\
\href{https://github.com/Denigmma/NewsRecSysBot}{NewsRecSysBot - demo}\\
Рекомендательная система новостей в Telegram: сбор контента из публичных каналов, онбординг через LLM-агента (Mistral) из свободного текста, кандидаты на эмбеддингах в Chroma (user→item, item→item, слабый user→user) + rerank Logistic MF по лайкам/дизлайкам и MMR-диверсификация.

\item \textbf{Внедрение архитектуры рекуррентных нейронных сетей GRU в библиотеку}\\ \href{https://github.com/Krekep/degann}{DEGANN}\\
    Работал с рекуррентными нейронными сетями в рамках пакета нейросетевой аппроксимации дифференциальных уравнений.


%\item \textbf{Image Processing Server} \\
%"Image processing server" – многопоточный сервер для обработки изображений, направленного на выполнение определенного
%спектра задач с возможностью одновременного использования множеством пользователей.
%Реализовано 4 функции для применения к изображению с возможностью дальнейшего расширения выполняемых задач. \\
%Перечень реализованных функций:
%\begin{itemize}
%    \item Наложение на изображение фильтра Гаусса
%    \item Выравнивание текста (под наклоном) на изображении
%    \item Поворот изображения на n градусов
%    \item Наложение черно-белого фильтра
%\end{itemize}
%Общение сервера и клиента происходит через TCP протокол.
%После подключения клиента сервер отправляет приветствие и просит выбрать желаемую операцию.
%После выбора операции пользователь загружает фотографию, над которой необходимо выполнить операцию,
%и в ответ от сервера получает обработанную фотографию. \\
%\textbf{Многопоточность:} Сервер способен параллельно обрабатывать запросы от разных подключенных клиентов одновременно.
%Когда новый клиент подключается, для него создается отдельный поток, который обрабатывает его запросы независимо от других клиентов.
%Сервер защищен от конфликтов при одновременном доступе к общим данным благодаря использованию mutex
%(синхронизация доступа к ресурсам сервера из разных потоков).\\
%Исходный код проекта доступен на GitHub:
%\href{https://github.com/Denigmma/Cpp-course/tree/main/Server_lab5}{Image Processing Server}.
\end{itemize}


\end{document}