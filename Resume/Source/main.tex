\documentclass[a4paper,14pt]{article}
\usepackage[utf8]{inputenc}
\usepackage[russian]{babel}
\usepackage{geometry}
\geometry{left=1in,right=1in,top=1in,bottom=1in}
\usepackage[colorlinks=true, linkcolor=blue, urlcolor=blue, citecolor=blue]{hyperref}
\usepackage{graphicx}
\usepackage{enumitem}
\usepackage{titlesec}
\usepackage{fontawesome5}
\usepackage{setspace}
%\usepackage{times}

\titleformat{\section}{\bfseries\large}{\thesection}{1em}{}
\titleformat{\subsection}{\bfseries\large}{\thesubsection}{1em}{}

\titlespacing{\section}{0pt}{0.5em}{0.5em}

\setlength{\parindent}{0pt}
\setlength{\parskip}{0.5em}
\setlist{nosep}

\begin{document}

% Заголовок резюме
\begin{center}
    {\Huge \bfseries Денис Мурадян Степанович} \\
    \vspace{0.5em}
    {\Large \bfseries Machine Learning and Software Engineer} \\
    \vspace{1em}
    \faPhone~ \href{tel:89052045723}{89052045723} \hspace{1em}
    \faEnvelope~ \href{mailto:muradyan.denis@inbox.ru}{muradyan.denis@inbox.ru} \hspace{1em}
    \faHome~ \href{https://yandex.ru/maps/-/CHRjvNpm}{Санкт-Петербург, Россия} \\
    \vspace{0.5em}
    \faGithub~ \href{https://github.com/Denigmma}{GitHub (Denigmma)} \hspace{1em}
    \faTelegram~ \href{https://t.me/DenisMuradyan}{Telegram (@DenisMuradyan)}
\end{center}

\section*{Hard Skills}
\hrule
\vspace{0.5em}
\begin{itemize}
    \item \textbf{Языки программирования:} Python, C, C++, SQL
    \item \textbf{Программное обеспечение:} Visual Studio Code, PyCharm, PgAdmin, Maple, VirtualBox, Cura, Tinkercad
    \item \textbf{Библиотеки и технологии:} NumPy, TensorFlow, PyTorch, Scikit-Learn, Jupyter, MatPlotLib, pandas, REST API, FastAPI, promt engineering, WSL2, Git, OpenCV, OpenACC, stb image, winsock2, Thread Pool, TCP, PostgreSQL
\end{itemize}

\section*{Soft Skills}
\hrule
\vspace{0.5em}
\begin{itemize}
    \item Умение быстро адаптироваться к новым технологиям и инструментам
    \item Эффективная работа в команде с развитым умением решать проблемы
    \item Развитые навыки коммуникации и активного слушания
    \item Умение управлять проектами, организовывать задачи и расставлять приоритеты
    \item Ответственность и внимание к деталям
    \item Постоянное стремление к профессиональному и личностному росту
\end{itemize}

\section*{Образование}
\hrule
\vspace{0.5em}
\begin{itemize}
    \item Окончил филиал московской частной школы "Столичный лицей" с углубленным изучением математики, физики и информатики.
    \item Бакалавриат СПбГУ Математико-механического факультета по направлению "Искусственный интеллект и наука о данных". Период обучения: 2023–2027.
\end{itemize}

\section*{Языки}
\hrule
\vspace{0.5em}
\begin{itemize}
    \item Русский (родной)
    \item Английский (Intermediate)
\end{itemize}

\section*{Олимпиады}
\hrule
\vspace{0.5em}
\begin{itemize}
    \item Призер 2 степени  Физико-Математической олимпиады МИЭТ 2023 по направлению математика
    \item Призер 3 степени в “Международная олимпиада Phystech.Internationals 2022” по направлению математика
    \item Призер 2 степени Олимпиады школьников Казахстанского филиала Московского государственного университета (МГУ) имени М.В. Ломоносова в 2022-2023 учебном году по математике
    \item Участник заключительного этапа олимпиады Petropolitan Science (Re)Search СПбГУ 2023-2024
    \item Участник заключительного этапа олимпиады "Я профессионал" по направлению "Искусственный интеллект" 2024-2025 \href{https://github.com/Denigmma/Practice_LogBook/blob/main/Yandex/Я%20профессионал%20(olimp)/2024-2025/certificates/ИИ(приглашение%20на%20заключительный%20этап).pdf}{certificates}
    \item Участник заключительного этапа олимпиады "Я профессионал" по направлению "Программирование и информационные технологии" 2024-2025 \href{https://github.com/Denigmma/Practice_LogBook/blob/main/Yandex/Я%20профессионал%20(olimp)/2024-2025/certificates/ПрогИнфТех(приглашение%20на%20заключительный%20этап).pdf}{certificates}
    \item Участник заключительного этапа олимпиады "Высшая лига" по направлению "Прикладная информатика и математика" 2024-2025
\end{itemize}

\section*{Соревнования - Хакатоны}
\hrule
\vspace{0.5em}
\begin{itemize}
    \item \textbf{Хакатон SPB Tech Run 2024 - 3 место} \\
    Разработка Интеллектуального чат-бота "Я Здесь Живу" (ЯЗЖ) на основе технологии Large Language Model (LLM) - для генерации ответов, с применением подхода Retrieval-Augmented Generation (RAG) - для эффективного поиска и использования релевантной информации из базы знаний. AI-помощник способен отвечать на разнообразные вопросы жителей о повседневной жизни в Санкт-Петербурге и его районах. Для предоставления ответов на повседневные вопросы, использует открытые данные города.

    \textbf{Сценарии вопросов (Agent Tools):}
\begin{itemize}
        \item Благоустройство, ЖКХ и уборка дорог
        \item Поиск контактов, основанный на Базе Контактов Санкт-Петербурга
        \item Поиск релевантной информации, ответ на вопрос основанный на Базе Знаний Санкт-Петербурга
        \item Информация об образовательных учреждениях (детские сады и школы)
        \item Раздельный сбор мусора
        \item Афиша мероприятий и красивые места города
        \item Поисковик на сайте \href{https://www.gov.spb.ru/}{Портала gov.spb.ru}
\end{itemize}
    Хакатон проводился с 30 ноября по 1 декабря 2024 года, в нём участвовало 80 человек. Наша команда заняла 3 место, продемонстрировав сильный акцент на практическую пользу, техническую проработку и возможность масштабирования проекта.
    Реализацию бота вы можете посмотреть на \href{https://github.com/Denigmma/Hackathon_SPBTechRun_ILiveHereBot}{GitHub (Hackathon SPBTechRun ILiveHereBot)}
\end{itemize}


\section*{Учебно-курсовые работы}
\hrule
\vspace{0.5em}
\begin{itemize}
\item \textbf{Внедрение архитектуры рекуррентных нейронных сетей GRU в библиотеку \href{https://github.com/Krekep/degann}{DEGANN}}

Курсовая работа посвящена реализации архитектуры Gated Recurrent Unit (GRU) на платформе TensorFlow для дальнейшей интеграции в библиотеку DEGANN, предназначенной для автоматического подбора топологии нейронных сетей, решающих задачи аппроксимации дифференциальных уравнений (ODE). Реализация GRU-модели должна улучшить точность и устойчивость решения задач с временными последовательностями и данными, зависящими от предыдущих состояний. В рамках работы проводится детальный сравнительный анализ путем разработки экспериментов на различных датасетах, моделирующих поведение дифференциальных уравнений.

\item \textbf{Реализация автоматизированной системы парсинга данных на основе больших языковых моделей \href{https://github.com/Denigmma/Automated_LLM_data_parsing_system}{Automated LLM data parsing system}}

\end{itemize}

\section*{Pet Projects}
\hrule
\vspace{0.5em}
\begin{itemize}

\item \textbf{Прогнозирование отказов оборудования в нефтегазовой промышленности} \\
Разработан для заключительного этапа студенческой олимпиады «Газпром» - конкурс проектов.
Проект представляет собой интеллектуальную систему мониторинга и прогнозирования аварийных ситуаций на
базе рекуррентной нейронной сети с архитектурой GRU. Модель анализирует временные ряды, поступающие с датчиков
(давление, температура, вибрация, расход газа и т.д.), и определяет вероятность сбоя или аварии.
Достигнуты высокие показатели качества: (accuracy = 0.9125, auc = 0.9777, precision = 0.8490, recall = 0.9178). \\[0.3em]
Система реализована на Python с использованием фреймворка FastAPI (REST API). В процессе работы данные нормализуются,
подаются на вход модели, а результаты прогнозов отображаются в реальном времени на встроенном dashboard.
Для удобства развёртывания предусмотрена возможность запуска проекта как локально, так и через LocalTunnel,
что обеспечивает доступ к сервису из сети Интернет.
Благодаря учёту временных взаимосвязей, высокой полноте (recall) и возможности оперативной визуализации система эффективно помогает
предотвращать потенциальные аварии, снижая риски и экономические потери на предприятиях нефтегазовой сферы. \\[0.3em]
Исходный код проекта доступен на GitHub:
\href{https://github.com/Denigmma/Gazprom_AI_analyzer_equipment_failure}{Gazprom AI analyzer equipment failure}.



\item \textbf{Image Processing Server} \\
"Image processing server" – многопоточный сервер для обработки изображений, направленного на выполнение определенного
спектра задач с возможностью одновременного использования множеством пользователей.
Реализовано 4 функции для применения к изображению с возможностью дальнейшего расширения выполняемых задач. \\
Перечень реализованных функций:
\begin{itemize}
    \item Наложение на изображение фильтра Гаусса
    \item Выравнивание текста (под наклоном) на изображении
    \item Поворот изображения на n градусов
    \item Наложение черно-белого фильтра
\end{itemize}
Общение сервера и клиента происходит через TCP протокол.
После подключения клиента сервер отправляет приветствие и просит выбрать желаемую операцию.
После выбора операции пользователь загружает фотографию, над которой необходимо выполнить операцию,
и в ответ от сервера получает обработанную фотографию. \\
\textbf{Многопоточность:} Сервер способен параллельно обрабатывать запросы от разных подключенных клиентов одновременно.
Когда новый клиент подключается, для него создается отдельный поток, который обрабатывает его запросы независимо от других клиентов.
Сервер защищен от конфликтов при одновременном доступе к общим данным благодаря использованию mutex
(синхронизация доступа к ресурсам сервера из разных потоков).\\
Исходный код проекта доступен на GitHub:
\href{https://github.com/Denigmma/Cpp-course/tree/main/Server_lab5}{Image Processing Server}.
\end{itemize}


\section*{Опыт и прочее проекты}
\hrule
\vspace{0.5em}
\begin{itemize}
    \item \textbf{Оптимизированное построение и визуализация графов:} Построение графов с алгоритмом Force-Directed Graph Drawing. Работал над визуализацией графов, минимизируя пересечения рёбер.
    \item \textbf{Детерминированный конечный автомат:}
    Реализация детерминированного конечного автомата (DFSM) для обработки строк.
    \item \textbf{Консольная игра (усложненная пошаговая игра):} Разработал усложненную версию классической ходилки с дополнительными заданиями и логикой шага на игровом поле.
    \item \textbf{Курс алгоритмов и структур данных:} Занимался профилированием кода, поиском утечек памяти.
    \item \textbf{Promt:} Работа с Promt запросами над нейронной сетью от компании "ARC Lab, Tencent PCG".
\end{itemize}

\end{document}